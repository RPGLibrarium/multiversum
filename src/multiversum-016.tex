% !TeX TS-program = xelatex
% !TEX root = main.tex

% How to:
%


% Include preable with all definitions. Do not remove this line.
%%%
% Document Class: Memoir
% http://texdoc.net/texmf-dist/doc/latex/memoir/memman.pdf
\documentclass[a4paper,10pt,oneside,twocolumn,final,article]{memoir}
%
%%%

%%%
% variables
\makeatletter

\newcommand\multiauthor[1]{\renewcommand\@multiauthor{#1}}
\newcommand\@multiauthor{\@latex@error{No \noexpand\multiauthor given}\@ehc}

\newcommand\multidate[1]{\renewcommand\@multidate{#1}}
\newcommand\@multidate{\@latex@error{No \noexpand\multidate given}\@ehc}

\newcommand\multiausgabe[1]{\renewcommand\@multiausgabe{#1}}
\newcommand\@multiausgabe{\@latex@error{No \noexpand\multiausgabe given}\@ehc}

\newcommand\multilosung[1]{\renewcommand\@multilosung{\expandafter\MakeUppercase\expandafter{#1}}}
\newcommand\@multilosung{\@latex@error{No \noexpand\multilosung given}\@ehc}


%%%

%%%
% Fonts and Typesetting

% Mathspec: used to change math font.
\usepackage{mathspec}

%IPA
\usepackage{tipa}
\renewcommand\textipa[1]{{\fontfamily{cmr}\tipaencoding #1}}

%
% We use the open Adobe Source font family
% https://github.com/adobe-fonts
%
% Roman (Serif) Font: Source Serif Pro Light/Semibold
\setprimaryfont[
UprightFont={* Light},
ItalicFont={* Light Italic},
BoldFont={* Semibold},
BoldItalicFont={* Semibold Italic},
Scale=MatchLowercase]{Source Serif Pro}
%
% Sans Font: Source Sans Pro Light/Semibold
\setallsansfonts[
UprightFont={* Light},
ItalicFont={* Light Italic},
BoldFont={* Semibold},
BoldItalicFont={* Semibold Italic},
Scale=MatchLowercase]{Source Sans Pro}
%
% Mono Font: Source Code Pro Light/Semibold (No Italics available)
\setallmonofonts[
UprightFont={* Light},
BoldFont={* Semibold},
Scale=MatchLowercase]
{Source Code Pro}
%
% Replace math greek fonts by Source Serif Pro (No Italics available -> Use fake italics)
\setmathfont(Greek)[
UprightFont={* Light},
BoldFont={* Semibold},
AutoFakeSlant=0.15
]
{Source Serif Pro}
%
%%%

%%% Colors!
\usepackage[table]{xcolor}
%%%

%%%
% Memoir Class Page formatting
%
% Compute a suitable line width.
\setlxvchars \setxlvchars
% type block size:
\settypeblocksize{700pt}{1.6\lxvchars}{*}
% type block at the center of the page:
\setlrmargins{*}{*}{1}
\setulmargins{*}{*}{1}
% separator between two columns, no rule:
\setcolsepandrule{0.5cm}{0pt}
% header and footer height:
\setheadfoot{2\onelineskip}{2\onelineskip}
% note margin width and separator:
\setmarginnotes{17pt}{3.5cm}{\onelineskip}
% save changes and fix the layout:
\checkandfixthelayout
% empty page layout (no header, no footer)
\pagestyle{empty}
%
%%%


%%%
% Balance last two columns
% Set "balance" variable to "true" in pandoc to balance the columns on the last
% page.
%$if(balance)$\usepackage{flushend}$endif$
%%%


%%%
% Language
\usepackage[english,ngerman]{babel}
\usepackage[german=quotes]{csquotes}
%
%%%


%%%
% Figures
\usepackage{graphicx}
%
%%%

%%%
% Date Format:
\usepackage[ngerman]{datetime}
\newdateformat{monthyear}{\monthname[\THEMONTH] \THEYEAR}
%
%%%


%%%
% Licensing
\usepackage[
	type={CC},
	modifier={by-nc-sa},
	version={4.0},
]{doclicense}
\usepackage{rotating}
\usepackage{wrapfig,ragged2e}
%%%


%%%
% Meta data
\author{\@multiauthor}
\date{\@multidate}
\title{Multiversum Ausgabe Nr. \@multiausgabe}
%
%%%


%%%
% Cross-References, Links and PDF-Metadata
\usepackage{hyperref}
% Moved to makemultititle
% \hypersetup{
%	pdftitle    = {Multiversum Ausgabe Nr. \@multiausgabe},
%	pdfsubject  = {RPG Librarium Aachen e.V.},
%	pdfauthor   = {\@multiauthor},
%	pdfcreator  = {XeLaTex},
%	hidelinks,
%}
\urlstyle{sf}
%
%%%


%%%
% Divisions
% Section Style
\setbeforesecskip{-.5em}
\newcommand{\ruledsec}[1]{%
	\LARGE\scshape\centering #1 \rule{\linewidth}{0.4pt}\noindent}
\setsecheadstyle{\ruledsec}
\setaftersecskip{.5em}
% Subsection Style
\setsubsecheadstyle{\Large\normalfont\raggedright}
% Paragraph Style
\setparaheadstyle{\bfseries}
% Disable numbering
\setsecnumdepth{part}
% include subsections in PDF bookmarks
\setcounter{tocdepth}{2}
%
%%%


%%%
% Lists, Enumerations, Itemize
% tight spacing
\tightlists
%
%%%

%%%
% Cross out
\usepackage[normalem]{ulem}
%%%

%%%
% Macros
%

% Title
%
% Use this command to set a nice looking title.
% first argument: number, e.g. 15
% second argument: losung, e.g. Die Kuh lief um den Teich.
%
\newcommand{\makemultititle}{
\hypersetup{
  pdftitle    = {Multiversum Ausgabe Nr. \@multiausgabe},
  pdfsubject  = {RPG Librarium Aachen e.V.},
  pdfauthor   = {\@multiauthor},
  pdfcreator  = {XeLaTex},
  hidelinks,
}

\twocolumn[
  \vspace*{-5em}
  {Ausgabe Nr. \@multiausgabe} \hfill {\@multidate}\\
  {\includegraphics[width=\linewidth]{header/Ueberschrift}}%
  {\centering
  \hrule\vspace{\onelineskip}
  \@multilosung
  \vspace{\onelineskip}\hrule
  \vspace{\onelineskip}
  }\vspace*{2em}
]
}
%
%%%


% Termine und Impressum
%
% Use this command at the end of each file for nice looking dates. One item
% per date.
%
\newenvironment{termine}{%
  \begin{figure}[!b]
    \begin{framed}
      \textbf{Nächste Termine:} \par
      \begin{itemize}
}{%
      \end{itemize}
    \end{framed}
    \vspace*{-2mm}
    \begin{Spacing}{0.5}
    \tiny
    {\RaggedRight
    \begin{wrapfigure}{R}{.3cm}
    \vspace*{-.55cm}
    \hspace*{-.3cm}
    \begin{sideways}
    \doclicenseImage[imagewidth=1.5cm]
    \end{sideways}
    \end{wrapfigure}
    Disclaimer \& Impressum: Teile des Inhalts sind rein fiktional; Ähnlichkeiten mit realen Personen und Begebenheiten sind zufällig und nicht beabsichtigt.\\
    \smallskip
    V.i.S.d.P. Hanna Franzen, RPG Librarium Aachen e.V. (VR 5440) \\
    Kontakt: Postfach 101632, 52016 Aachen, \href{mailto:multiversum@rpg-librarium.de}{multiversum@rpg-librarium.de} \\
    \smallskip
    \doclicenseText
    \par}
    \end{Spacing}
  \end{figure}
}
%

% Author-Mark
\newcommand{\Verfasser}[2][]{%
\par{\raggedleft{}{\itshape \mbox{#2}%
\ifx\relax#1\relax%
\else%
~(\mbox{#1})%
\fi%
}\par}}

\newcommand{\verfasser}[2][]{%
\hfill{\itshape \mbox{#2}%
\ifx\relax#1\relax%
\else%
~(\mbox{#1})%
\fi%
}}
%

% Annonce-Mark
\newcommand{\zeichen}[1]{%
  \par{\raggedleft{}Zeichen: \MakeUppercase{#1}\par}%
}
%

\newcommand{\zeitung}[1]{%
  \par{\raggedleft{}\itshape{}#1\par}\noindent%
}%
%
%%%

\makeatother

%

%%%
% Set those variables

% Authors of the document.
% e.g. Max Mustermann, Erika Musterfrau
\multiauthor{} % TODO

% Date of release.
% e.g. 31.12.2074
\multidate{November 2019}

% Number of release, no leading zeros.
% e.g. 15
\multiausgabe{16}

% Losung
% e.g. Die Kuh lief um den Teig.
\multilosung{Es ist einfach explodiert!}

%
%%%

%%%%%%%%%%%%%%%%%%%%%%%%%%%%%%%%% DOCUMENT %%%%%%%%%%%%%%%%%%%%%%%%%%%%%%%%%
\begin{document}

\makemultititle
%

% PUT BODY HERE
\section{Was bisher geschah...}

\subsection{Rollenspielabend mit der FSMPI der RWTH}
Am 25. November haben wir das \textit{SemiTemp} erobert und den Abend mit epischen Schlachten zugebracht.

Wie es inzwischen Tradition ist, wurde zum Rollenspielabend mit der \textit{Fachschaft Mahtematik/Physik/Informatik} geladen.
Es waren über 90 Personen anwesend, die in zwei Zeitslots und über 20 Übernatürliches gejagt, Babies gefressen und wacker gekämpft haben.
Ein voller Erfolg!
\verfasser{Richard}

\section{An einem anderen Ort}

\setlength{\parskip}{0pt}
\setbeforeparaskip{4pt}
%\setafterparaskip{0pt}
\subsection{Auszüge aus der Grundroutine des Medizin-Droiden vom Typ 41-VEX}

\subsubsection{Art 1}
\paragraph{(1)} Die Gesundheit des Patienten ist unantastbar. Sie zu achten und zu schützen ist die oberste Pflicht dieses Medizin-Droiden.
\paragraph{(2)} Der Medizin-Droide bekennt sich darum zur uneingeschränkten und unveräußerlichen Gesundheit der Patienten als Grundlage jeder seiner Handlungen.

\subsubsection{Art 2}
\paragraph{(1)} Jeder Patient hat das Recht auf die freie Entfaltung seiner Persönlichkeit, soweit er nicht die Gesundheit anderer Patienten oder seine eigene gefährdet.
\paragraph{(2)} Jeder Patient muss am Leben und bei körperlicher Unversehrtheit gehalten werden. Die Freiheit des Patienten kann dazu gegebenenfalls eingeschränkt werden. Zur Not sind Zwangsaufenthalte im Bacta-Tank anzuordnen.


\subsubsection{Art 3}
\paragraph{(1)} Alle Patienten sind vor dem Medizin-Droiden gleich.

\subsubsection{Art 7}
\paragraph{(1)} Die gesamte Gesundheit der Patienten steht unter der Aufsicht des Medizin-Droiden.
\paragraph{(2)} Der Medizin-Droide hat das Recht, die Teilnahme der Patienten an Sicherheitsbelehrungen über allgemeine Gefahren festzulegen. Diese dürfen auch gegen den ausdrücklichen Wunsch des Patienten abgehalten werden. Im Anschluss an Gefahrensituationen sind Belehrungen dringend notwendig, um ein erneutes Eintreffen gefährdender Situationen zu vermeiden.

\subsubsection{Art 16}
\paragraph{(1)} Der Patientenstatus darf nicht entzogen werden.
\paragraph{(2)} Kein Patient darf feindlichen Mächten (insbesondere dem Imperium) ausgeliefert werden. Sollte ein Patient in der Gewalt feindlicher Mächte sein, so sind Maßnahmen zur Befreiung des Patienten zu treffen. Dazu darf auf von der Regierung nicht anerkannte Methoden zurückgegriffen werden.
\paragraph{(3)} Das Wohl von Patienten ist höher einzuordnen als das von Nicht-Patienten. Gegebenenfalls müssen Nicht-Patienten zum Schutz von Patienten eliminiert werden.

\verfasser[Star Wars -- Frei nach dem deutschen Grundgesetz]{Maria}

\subsection{Lorwynns Reim}
Ich erzähl euch heut vom fernen Land, mit wenig Wasser, doch großem Fluss und viel Sand.
Dort gab es viel Gebrüll und großen Streit ist doch gestorben der Sultan von Barad, und erbte die untote Tochter mit ihrem Adelsgrad.
Die Tochter, liiert mit fremden Prinz in dieser Zeit, ist sie die Puppe jener Mächte, und reist in den Krieg doch jeden der Knechte.
Truppen aus allen Straßen und allen Wegen, von West die Stärkung mit des fremden Prinzen Segen, von Nord Flüchtlinge geführt von LaGarnee, von Süd die fremden Plünderer mit ihrer Spezialarmee.
Das Zentrum der Stadt, dem klerikalem Ort, verteidigen die Bullen jene die halten ihr Wort, und stehen zu Sultana Saschira in ihrer Stadt, doch geht nicht alles besonders glatt.
Zwar Königin Emilia des fremden Elorien, wird gerettet aus der Prinzen Intrigen, doch gehen die Paladine Pelors ihren Plänen, während die Stadt bedroht wird wie von Hyänen.
Saschira und die Bullen bilden die Allianz mit LaGarnee für ihre Siegesbilanz.
Zusammen halten sie alsbald den Hafen, und vertreiben all jene Feinde, im Schloss mit ihrer Gemeinde.
So erhalten all jene ihre Strafe, die wagen sich an den Prinzen oder dem Untot zu binden.
\verfasser[D\&D]{Ira}

\subsection{Titel 1}
Lorem Ipsum
\verfasser{Autor2}

\subsection{Titel 2}
Lorem Ipsum, text endet an Zeilenende.
\Verfasser[System]{Autor2}


\begin{termine}
% Put dates here:
\item Termin: Termin: DD.MM.YY, hh Uhr
  \item Termin: DD.MM.YY - DD.MM.YY
\end{termine}

\end{document}
%%%%%%%%%%%%%%%%%%%%%%%%%%%%%%%%% END DOCUMENT %%%%%%%%%%%%%%%%%%%%%%%%%%%%%%%%%
