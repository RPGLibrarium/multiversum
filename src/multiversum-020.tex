% !TeX TS-program = xelatex
% !TEX root = main.tex

% How to:
%


% Load class with all definitions. Do not remove this line.
% Options will be passed to Memoir
\documentclass[final]{multiversum}
%

%%%
% Set those variables

% Authors of the document.
% e.g. Max Mustermann, Erika Musterfrau
\multiauthor{Paul, Hanna}

% Date of release.
% e.g. 31.12.2074
\multidate{}

% Number of release, no leading zeros.
% e.g. 15
\multiausgabe{20}

% Losung
% e.g. Die Kuh lief um den Teig.
\multilosung{Immerhin bist du kein Pferd, sonst wärst du jetzt tot.}


% Logo
% Use a different logo. Defaults to Ueberschrift.svg
%\multilogo{Ueberschrift_xmas}

%
%%%

%%%%%%%%%%%%%%%%%%%%%%%%%%%%%%%%% DOCUMENT %%%%%%%%%%%%%%%%%%%%%%%%%%%%%%%%%
\begin{document}

\makemultititle
%

% PUT BODY HERE
\section{Was bisher geschah...}

\subsection{In der Redaktion}
Montag morgen, 8 Uhr. 
Ich komme in die Redaktion des Multiversums, in der Hand einen Coffee to go.
Huch. Niemand hier? *gähn*
Ich schaue auf die Uhr, Zeit und Datum stimmen.
Alles ist still und wirkt verlassen.
Heute sollte ich zum Jahresbeginn auch mein neues Praktikum beginnen.
Es hieß, ein anderer Praktikant solle mich empfangen und einführen, während die Belegschaft auf einem alljährigen Ausflug in die Multiversen ist.
In Rittersäälen speisen wäre mir jetzt auch lieber, aber man braucht ja was für den Lebenslauf.
Der Andere ist bestimmt zu spät oder hat mich vergessen.
Professionell betrachte ich den nahezu leeren Eingangskorb.
Darin sind ein Angebot für eine Genitalverlängerung (zum Papierkorb) und eine Rechnung (in den Korb der Verwaltung).
Der Kaffee wird aufgesetzt.
Jetzt müsste mich jemand weiter anleiten.
Unschlüssig setze ich mich an einen zufälligen Schreibtisch und lasse meinen Blick darüber schweifen.
Zwei Monitore, eine Tastatur und Maus, ein Foto eines Loveinterests, ein verdorrter Kaktus.
Über allem liegt eine fingerdicke Staubschicht.
Puh, wo sind die denn alle?
Ich öffne unbeobachtet die Schreibtischschubladen, finde aber keine Hinweise.
Auf einem Notizzettel steht: \"Erinnert mich daran, was über Sirenen zu schreiben. Sirenen singen nur beruflich, nicht in ihrer Freizeit.\"
Ich schüttel den Kopf, das ist nicht hilfreich.
Ich stehe wieder auf und streife durch das verlassene Gebäude.
Hinter dem großen Büro mit mehreren Schreibtischen liegt eine Teeküche, jetzt mit fertigem Kaffee.
Daneben gibt es eine Toilette und eine sehr kleine Tür mit einer Maus darauf.
Etwas weiter im Gang, hinter ein paar einzelnen Büros, scheint Licht zu brennen.
Ich halte mich etwas gerader, da ist also doch jemand.
Als ich das Licht erreiche, strahlt es aus einem gläsernen Sitzungsraum.
Über einem langen Holztisch ist ein kleiner Ring auf einem Dreifuß aufgebarrt, aus dem das Licht strömt und das Zimmer erhellt.
Neugierig öffne ich die Schiebetür zum Raum, wobei mir die Tintenflecken auffallen, die das Glas von innen verunstalten.
Um den Tisch herum liegen knitterige Papiere verteilt und mein Haar wird von einem Windstoß zerzaust, der durch den Raum fegt.
Als ich mich dem Licht nähern will, zuckt neben mit ein violetter Blitz auf, durch den ein Jüngling seinen Kopf steckt.

"Mist, ist hier richtig?"

Er schaut sich um, erbleicht, als er mich sieht, und streckt die Hand nach mir aus.

"Bist du die neue Praktikantin? Ich soll dich eigentlich einführen, aber ich hab Scheiße gebaut. Komm, wir müssen das fixen, bevor es jemand merkt!"

Zögerlich strecke ich ihm die Hand entgegen, da ergreift er sie schon und zieht mich durch das Weltenportal.

\verfasser{Die Praktikantin}

\subsection{Workshop: ABC}
Lorem Ipsum
\verfasser{Autor1}



\subsection{Finde dein Weg, mit einer Pathfinder Runde}
Gruppe von 4 suchen Spieler*innen, mit Interesse für Rollenspiel, für gemütliche Pathfinder Abende zusammen.
Wir sind bilingual, offen für eine neue (schwebende) Welt.
Zeichen: Revolution
\zeitung{5.4.-311}
\verfasser[Pathfinder]{Paul}

\section{An einem anderen Ort}

\subsection{Herbert @herbwest}
That moment. Als Tourist durch Innsmouth kommen und merken, dass die eigene Familie da her kommt. \#deep
\zeitung{5:12am $\cdot$ 23th May 1929}
\verfasser[Cthulhu]{Hanna}

\subsection{Firenzes Panoptikum fabuloser Tinkturen}
Hühneraugen, Warzenweyden, Schmerzensreygen\\
Firenze Fabuloso heilt so ziemlich jedes Leyden!\\
Liebesleyd und kalter Schart oder sogar Alkehyst,\\
Firenze Faboloso weiß wie man's bemisst!\\[1em]

Extravagante Tinkturen zu annehmbaren Preisen.\\
Bald in Ihrer Stadt, nur für kurze Zeit! 
\Verfasser[DSA5]{Frieder}


\begin{termine}
% Put dates here:
\item Monatliches Treffen: 16.01.2021, 19 Uhr
\item Weltenbrücken: 30.01.2021, 14-18 Uhr
\end{termine}
\impressum

\end{document}
%%%%%%%%%%%%%%%%%%%%%%%%%%%%%%%%% END DOCUMENT %%%%%%%%%%%%%%%%%%%%%%%%%%%%%%%%%
