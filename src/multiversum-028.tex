% !TeX TS-program = xelatex
% !TEX root = main.tex

% How to:
%


% Load class with all definitions. Do not remove this line.
% Options will be passed to Memoir
\documentclass[final]{multiversum}
%

%%%
% Set those variables

% Authors of the document.
% e.g. Max Mustermann, Erika Musterfrau
\multiauthor{Franca, Sebi, Henri, Konstantin}

% Date of release.
% e.g. 31.12.2074
\multidate{01.07.2024}

% Number of release, no leading zeros.
% e.g. 15
\multiausgabe{28}

% Losung
% e.g. Die Kuh lief um den Teig.
\multilosung{Wenn Käse alt wird, läuft er schneller; bei Menschen ist es umgekehrt.}


% Logo
% Use a different logo. Defaults to Ueberschrift.svg
%\multilogo{Ueberschrift_xmas}

%
%%%

%%%%%%%%%%%%%%%%%%%%%%%%%%%%%%%%% DOCUMENT %%%%%%%%%%%%%%%%%%%%%%%%%%%%%%%%%
\begin{document}

\makemultititle
%

% PUT BODY HERE
\subsection{Open Source Software $\Phi$-LEAS-1}
Ahoi Chummer,\\
ich hab ein neues Bastelprojekt! Proudly presenting...

\textbf{$\Phi$-LE4S-1}: \textbf{P}acific \textbf{HI}gh \textbf{L}eadership
\textbf{E}ngine \textbf{4} \textbf{S}eafaring 1.0 (gesprochen: \textit{Phileas
One}).

$\Phi$-LE4S-1 ist eine Open Source Software zur Steuerung von Motor- und
Segelyachten. Der Agent kann Navigations-, Logistik- und Steuerungsaufgaben
übernehmen. Unter anderem checkt $\Phi$-LE4S-1 die Wetterlage, errechnet die
übliche Höhe von Schmiergeldern in lokalen Häfen,  erstellt euch Shanty-Cover
eurer Lieblingssongs und spielt Quartiermeister (ein fancy Name dafür, dass
$\Phi$-LE4S-1 sagt, wenn das Klopapier alle ist).

Die Software befindet sich noch in der Entwicklung. Ich freue mich über Feedback!

LG, November

\bigskip

Edit:\\
Aufgrund wiederholter Nachfrage: $\Phi$-LE4S-1 ist benannt nach Phileas Fogg,
dem Protagonisten aus Jules Vernes \enquote{In 80 Tagen um die Welt}.

\noindent [eBook Download][Audiobook Download]\\
\noindent [Matrix-Game Download]

\bigskip

---PATCH NOTES---\\
Additions
\begin{itemize}
    \item \enquote{Talk like a Pirate} Sprachpaket für $\Phi$-LE4S-1 installiert
    \item Neuer Nebelhorn-Sound \enquote{Foggwulf}: Schauriges Wolfsgeheul für neblige
    Nächte auf See
    \item Einparken vereinfacht durch Qualitätsvergleich-Automanöver (QUAM)
\end{itemize}

\bigskip

\noindent Bugfixes
\begin{itemize}
    \item Keine GPS-Konflikte mehr durch \enquote{Portal Magic Glitch} und \enquote{Mysterious Teleport Dog}
    \item Erste-Hilfe-Sets mit Software \enquote{m4gr0n-X} haben jetzt keine Angst mehr
    vor Wasser
    \item $\Phi$-LE4S-1 interpretiert Bogenschützen nicht mehr als
    \enquote{overpowered}, außer, das Datum des Bordcomputers wird manuell ins 14.
    Jahrhundert gesetzt
    \item $\Phi$-LE4S-1 schickt keine \enquote{Bei Swafnir!}-Funksprüche mehr an die
    Hafenbehörden
    \item \enquote{Infinite Cheese Glitch} am Sojafabrikator behoben
    (unaufhaltsame Produktion von Sojabrei mit Käsegeschmack in der Kombüse)
    \item \enquote{Ohm Folker}-Bug ist jetzt ein Feature ($\Phi$-LE4S-1 funkt
    ungefragt andere Schiffe an, um Wikingerballaden vorzutragen)
\end{itemize} 
\Verfasser[Shadowrun]{Franca}

\subsection{Gestriger Jahrhundert-Überfall auf die Hauptfiliale der OLJ-Bank mit sechs Toten und fünf Schwerverletzten}
Am gestrigen Tage ereignete sich ein spektakulärer Überfall auf die Hauptfiliale
der OLJ-Bank in Kirchen. Unbekannte infiltrierten die Filiale und verschafften
sich Zutritt zum Tresor mit allen Bankschließfächern. Der Tathergang bleibt
zu weiten Teilen unklar, scheint allerdings von langer Hand her geplant
gewesen zu sein. Die Räuber, es werden drei bis fünf externe und eine interne Komplizin
vermutet, betraten die Filiale mit scheinbar unterschiedlichsten Intentionen und
verschafften sich so auf drei Weisen Eintritt in die hinteren Räume der Filiale.

Ein Tatverdächtiger verschaffte sich mit einer einfachen Währungswechselanfrage
Zutritt zur Bank und beschäftige damit die vermutete interne Komplizin für
mehrere Stunden. Diese scheint sich, bis zum jetzigen Standpunkt der
Ermittlungen, erst zur Tatzeit den Räubern angeschlossen zu haben und dafür das
Geld des Währungswechsels als Anteil erhalten zu haben. Von der 46-jährigen
Mutter zweier Kinder fehlt seit dem Abend jede Spur, allerdings verdichten sich
die Hinweise auf eine Flucht aus dem Land in Richtung Panama.

Während die Ablenkung durch die Insiderin stattfand, verschafften sich zwei als
Geldtransporterfahrer gekleidete Unbekannte durch Social-Engineering-Taktiken
Zutritt in die hinteren Räume der Bank; dort sollen sie schließlich Verkleidungen
gewechselt haben, um unerkannt zu bleiben und sich Zugang zum Sicherheitssystem
zu verschaffen. Die zuvor gezielt angesprochene Bankkauffrau (36 W), welche
weiterhin über die Unzulänglichkeiten der Kollegen und ihrer miserablen
Situation klagt, steht zum jetzigen Zeitpunkt unter Mordverdacht. Die
Angestellte, welche schon vorher durch aggressives Verhalten aufgefallen war,
scheint von den Räubern unter erheblichen psychischen Stress gesetzt worden zu
sein, sodass diese einen der Räuber physisch attackierte und sich dabei selbst
leicht verletzte. Als anschließend einer ihrer Kollegen (22 M) sie zu beruhigen
versuchte, attackierte und tötete sie diesen mit 12 Stichen einer Kuchengabel.
Momentan wird die Verdächtige auf ihrer Zurechnungsfähigkeit hin untersucht und
kommenden Montag einem Strafrichter vorgeführt.

Als vermutete dritte Säule des Angriffs wurde der Geburtstag des Filialbesitzers
(68 M) als Ablenkung inszeniert, wobei dieser ums Leben kam. Augenzeugen
berichteten, dass auf der Party ungewöhnlich viele Frauen vertreten waren,
welche scheinbar eine enge Verbindung zum Besitzer zu pflegen schienen. Die
Situation eskalierte laut Augenzeugen schlagartig, als sich herausstellte, dass
der Besitzer Affären mit einigen dieser Frauen geführt hatte. Als den
Frauen sowie der Ehegefährtin des Verstorbenen dies klar wurde, entbrannte
eine Massenschlägerei, bei der der Besitzer und eine weitere Frau noch am Tatort
ihren Verletzungen erlagen. Weitere 4 Gäste wurden bei der Schlägerei schwer
verletzt. Bisher unklar ist den ermittelnden Kommissaren der Einfluss der
scheinbar stark mit Kokain versetzten Geburtstagstorte des Verstorbenen. Nach
ärztlichem Befund hat die Aufregung der Schlägerei in Kombination mit der
präparierten Geburtstagstorte weiteren zwei Gäste (77 M und 86 W) ihr Leben
gekostet. Den Ermittlern ist bisher nicht klar, woher die Torte kam und wer die
weiteren Gäste einlud.

Auf diese 3 Weisen verschafften sich die Unbekannten zusammen Zutritt zum Keller
der Filiale und deaktivierten dort das scheinbar zuvor schon seit Wochen defekte
Alarmsystem gänzlich. Das dort zugeteilte Sicherheitspersonal (55 M) wurde erst
durch Social-Engineering-Taktiken von ihrem Arbeitsplatz entfernt und
anschließend durch physische Gewalt vorübergehend ausgeschaltet. Der ärztliche
Befund zeigt ein schweres Schädel-Hirn-Trauma, allerdings ist der Betroffene auf
dem Weg der Genesung.

Aus dem Tresor wurden der Inhalt diverser Bankschließfächer und erhebliche Mengen
Bargeld entwendet und scheinbar über einen Zugang auf dem Dach der Filiale
abtransportiert. Die ansässige Polizei sucht weiterhin nach Zeugen zum gestrigen
Tathergang und Hinweisen zu den Unbekannten. Ebenfalls sind Hinweise zu dem
Verbleib der 46-jährigen Bankangestellten erwünscht. 
\Verfasser[One Last Job]{Sebi}



\subsection{Kurzvorstellung: \enquote{One Last Job}}
Das Micro-Rollenspielsystem \enquote{One Last Job} ist ein Hack des etwas bekannteren
Systems \enquote{Lasers and Feelings}. Die Kernidee ist enorm simpel: Eure Gruppe von
ca. 3-6 Spielenden plant, einen Überfall historischen Ausmaßes auf ein vorher
bestimmtes Ziel durchzuführen. Mehr Vorbereitung als die Auswahl des Ziels
ist nicht nötig, dank der Kernidee des Systems: Alles ist Teil des Plans.

Konkret könnt ihr als Spieler zu jeder Zeit versuchen, die (Vor-)geschichte neu
zu schreiben, indem ihr euch ausdenkt, welche Vorbereitungen bereits in der
Vergangenheit getroffen wurden: Wurde die Wache, die euch gerade festnimmt,
bereits bestochen? Fällt der stille Alarm, den ihr auslöst, gar nicht auf, da ihr
dafür gesorgt habt, dass dieser bereits die letzten zwei Wochen konsistent zur
gleichen Zeit ausgelöst wurde und ein \enquote{Techniker ist informiert}-Schild an den
Warnleuchten hängt? War in der Geburtstagstorte für die Geschäftsführung von
Anfang an ein Tresorknacker eingebacken? Wenn ihr erfolgreich darauf würfelt,
ist eure Idee jetzt offiziell Teil der Geschichte. Außerdem dürft ihr
anschließend der Spielleitung Fragen über die Welt stellen, die wahrheitsgemäß
beantwortet werden müssen und auch offiziell Teil der Geschichte werden. Ebenso
flexibel ist das Inventarsystem: Ihr startet mit Inventarplatz, nicht mit
Gegenständen. Falls ihr einen Gegenstand braucht, habt ihr diesen offensichtlich
von Anfang an mitgenommen, sofern noch genug Kapazität übrig ist.

Das Spiel ist auf einen Oneshot ausgelegt, ihr solltet ca. 2-4 Stunden
mitbringen, je nachdem, wie viel kreativen Blödsinn sich eure Runde ausdenkt, der
noch in der Story verarbeitet werden möchte. Rollenspielvorkenntnisse braucht
ihr als Spielgruppe keine; es hilft jedoch, wenn ihr einen Hang dazu habt,
kreativen (und für die Story einigermaßen konstruktiven) Schabernack zu treiben.
Die Spielleitung sollte sich darauf einlassen können, den Unsinn der Spieler in
eine Gesamtgeschichte zu weben. In der Praxis ist das aber einfacher, als es
vielleicht zuerst klingt.
\verfasser{Henri}

\vfill

\section{An einem anderen Ort\ldots}

\includegraphics[width=0.48\textwidth]{src/img/RADIOAKTIVE_ARBEITSKRAFT_greyscale.png}
\Verfasser[LARP]{Konstantin}

\vfill

\begin{termine}
% Put dates here:
\item Monatstreffen: 16.07.24\\(bei gutem Wetter im Westpark)
\end{termine}
\impressum

\end{document}
%%%%%%%%%%%%%%%%%%%%%%%%%%%%%%%%% END DOCUMENT %%%%%%%%%%%%%%%%%%%%%%%%%%%%%%%%%
