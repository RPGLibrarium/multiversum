% !TeX TS-program = xelatex
% !TEX root = main.tex

% How to:
%


% Load class with all definitions. Do not remove this line.
% Options will be passed to Memoir
\documentclass[final]{multiversum}
%

%%%
% Set those variables

% Authors of the document.
% e.g. Max Mustermann, Erika Musterfrau
\multiauthor{Konstantin, Franca, Hanna}

% Date of release.
% e.g. 31.12.2074
\multidate{März 2022}

% Number of release, no leading zeros.
% e.g. 15
\multiausgabe{24}

% Losung
% e.g. Die Kuh lief um den Teig.
\multilosung{Buntstifte sind auch nur farbenfrohe Holzpflöcke}


% Logo
% Use a different logo. Defaults to Ueberschrift.svg
%\multilogo{Ueberschrift_xmas}

%
%%%

%%%%%%%%%%%%%%%%%%%%%%%%%%%%%%%%% DOCUMENT %%%%%%%%%%%%%%%%%%%%%%%%%%%%%%%%%
\begin{document}

\makemultititle
%

% PUT BODY HERE
%\section{Was bisher geschah...}

\subsection{Multiversum sucht die Super-Belagerungswaffe}
Kennt du das auch? 
Die Helden (ein Begriff, den wir hier sehr lose verwenden\footnote{Eine gekonnte Referenz auf Kobolde! Super-Delüx-Ausgabe, Pegasus Press, 2006, Seite 4, Paragraph 4, Zeile 4. Duh.})
 sind in einer befestigten Stadt, die leider gerade belagert wird?
Du, als Spielleiter von Welt, hast natürlich schon die große Schlacht in der Planung.
Nur, wie kommt das angreifende Heer durch die Mauern, damit deine feigen, den Ausfall scheuenden Helden sich kloppen können?
Also baust du das Übliche ein: Leitern und fahrbare Türme und Trebusch\dots{} Trebuk\dots{} Katapulte halt.

Aber\dots{} Moment mal! Das sieht doch jeder Gurkenheld kommen!
Damit auch du mehr Vielfalt in deine Belagerungsgeräte bringen kannst, findest du auf der Rückseite eine praktische Würfeltabelle!
\verfasser{Hanna}

\section{An einem anderen Ort}

\subsection{Aufregung um HoneyCon - Container explodiert}
\textit{Schlüsselfeld}. Eine Explosion erschütterte am vergangenen Wochenende das Gelände der HoneyCon-Messe in Schlüsselfeld. 
Auf bislang ungeklärte Art explodierte ein Honigcontainer gegen 12 Uhr mittags. 
Wie durch ein Wunder wurde niemand verletzt - allerdings wurden in der Nähe zwei bewusstlose Grizzlybären aufgefunden, die wohl durch die Explosion betäubt worden waren.

Augenzeugen zufolge wurden am Tatort mehrere Personen gesehen, die in Bärenkostümen gesteckt haben sollen. 
Diese verließen den Container überstürzt und nutzten danach einen geklauten Kompaktwagen als Fluchtfahrzeug. 
Die Redaktion erreichte außerdem ein Bekennerschreiben von einer offenbar ökoextremistischen Organisation, die sich \enquote{Die Flotten Bienen} nennt. 
Als Grund des Attentats werden \enquote{zahlreiche Verbärchen gegen die Bienheit} genannt.

Der Betreiber des Events, Bruno Medvedev, konnte bis zum Redaktionsschluss nicht für einen Kommentar erreicht werden.

Die HoneyCon findet in Schlüsselfeld seit 2017 jährlich statt. 
Nachdem sie von Grant Howitt ins Leben gerufen wurde, übernahm Bruno Medvedev unter mysteriösen Umständen schon bald die Leitung des Events. 
Der Unternehmer führt sein Leben sehr privat und wurde noch nie in der Öffentlichkeit gesehen. 
Höhepunkt der HoneyCon ist die große Honigprobe, bei der große Mengen Honig an die Besuchenden der Messe ausgeteilt werden. 
Dieser Honigvorrat war Unternehmensangaben zufolge der, der bei der Explosion vernichtet wurde.
\Verfasser[Honey Heist]{Konstantin}

\subsection{Beitrag im magischen Rundfunk, ??.??.1973}
Große Erleichterung in Stratford-upon-Avon: die 10-jährige Melinda Carmichael ist wieder aufgetaucht. 
Das Mädchen aus einer Familie mit einer muggelstämmigen Mutter war die letzten zwei Tage vermisst worden. 
Die Ängste, Melinda sei von Todessern entführt worden, haben sich offenbar nicht bewahrheitet - sie habe sich laut der Familie in einer Höhle versteckt. 
Mr. Carmichael sagte: \enquote{Ich bin zutiefst erleichtert, dass meine Tochter wieder da ist. 
Wir werden uns erstmal einen langen Urlaub in den Staaten gönnen, um über den Schreck hinwegzukommen.}
Mr. Carmichael hatte letzten Monat eine Petition gestartet, um das Etat des Aurorenbüros zu erhöhen. 
Wir wünschen ihm und seiner Tochter einen schönen Urlaub. 
\verfasser[Wizarding-World-LARP]{Konstantin}

\begin{table*}[!t]
      \begin{framed}
      \begin{tabular}{p{0.05\textwidth}p{0.17\textwidth}p{0.72\textwidth}}
      \textbf{1W15} & \textbf{Belagerungsgerät} & \textbf{Wozu brauche ich das Ding?}\\
      1             & Rammbock/Widder           & Zum Öffnen von Türen/ Toren oder Zerstören von z.B. Alabasterschnitzereinen. \\
      2             & Petarde                   & Wie ein Rammbock, aber mit mehr Krach.\\
      3             & Drache                    & Gut für Flugangriffe (Brandbomben, Gezielte Extraktion, Tod und Verderben).\\
      4             & Griechisches Feuer        & Jap, im Mittelalterlicher hatten die auch schon Flammenwerfer.\\
      5             & Katapult                  & Falls dir das Wort \enquote{Torsionsgeschütz} entfallen ist.\\
      6             & Torsionsgeschütz          & Macht Dinge schnell, die dann Schaden an Mauern und Leben anrichten.\\
      7             & Blide/Onager              & Fr. Trébuchet. Kann man einfacher hektisch schreien als \enquote{Torsionsgeschütz}.\\
      8             & Kanone                    & Kann auch Dinge schnell und aua machen, aber mit mehr Krach.\\
      9             & Balliste/Skorpion         & Falls du eine wirklich große Armbrust brauchst.\\
      10            & Mauerbohrer               & Falls du einen wirklich großen Bohrer brauchst.\\
      11            & Sturmleiter               & Für einfallende Orks und Feuerwehrleute.\\
      12            & Belagerungsturm           & Kippt weniger schnell um als eine Sturmleiter.\\
      13            & Katze                     & Hütte zum Anschleichen. Am Besten vorher mit frischem Mist beschmieren.\\
      14            & Tonnelon                  & Lifehack für Highground: Bogenschützen in die Tonne, Tonne hoch.\\
      15            & Antwerk                   & Falls du dir gar nichts ausdenken möchtest, aber gebildet klingen willst.\\

      \end{tabular}
      \end{framed}
      \end{table*}

%\section{Werbung}
\subsection{Der Staubsaugerdroide, den Sie suchen!}

Die Hygieneroutinen des Staubsaugerdroiden MAW lassen die Prozessoren jedes Medidroiden höher schlagen!
Lästiger Asteroidenstaub im Frachtraum?
Verkohlte Asche von ungebetenen Besuchern auf der Laderampe?
Katzenhaare im Schmugglerversteck? Wookieepelz im Crewquartier?\\
\begin{center}\textsc{Kein Problem für MAW mit seiner antiallergenen Feinstaubfiltertechnologie nach modernsten Coruscanter Standards!}\\\end{center}
Bacta-Spritzer im Medizinraum?
Verschütteter Ryloth-Schnaps in der Küche?
Blutspritzer von Freund und Feind im Cockpit?\\
\begin{center}\textsc{Die auf Mon Cala entwickelte Flüssigkeitssaugfunktion wird auch damit fertig!}\\\end{center}
Ryll Spice im Frachtraum?
Überreste explodierter Rodianer?
Die kulinarischen Experimente des Piloten?
\begin{center}\textsc{Innerhalb von Minuten kann MAW jedes Beweisstück verschwinden lassen!}\\\end{center}
Ausgerüstet mit einem neuartigen Navigationssystem, einem Adapter für alle herkömmlichen Hyperraumklappen, langanhaltendem Akku, großer beutelfreier Saugkapazität und parkettschonenden Rollen ist MAW bereit, jedes versiffte Raumschiff wieder auf Vordermann zu bringen.
\begin{center}\textsc{Dieser kleine Staubsaugerdroide würde selbst Tatooine im Handumdrehen sauber bekommen!}\\\end{center}
Bestellen Sie noch heute den Staubsaugerdroiden MAW für nur \textsc{2000 Credits} und erhalten Sie wahlweise einen Blaster oder einen modischen Hut zu Ihrer Bestellung dazu. 
Bestellungen holografisch oder per Textnachricht an Ra'an Thashin.\\
%\begin{center}\textsc{*Kundenstimmen*}\end{center}

\textsc{Kundenstimmen}
\begin{itemize}
\item \enquote{MAW versteht sich ausgezeichnet mit MINI, unserem Flammenwerfer-Droideka.
      MAW saugt ohne Murren die Asche hinter MINI auf und die zwei sind direkt unzertrennlich geworden. 
      Es ist schön, dass unser MINI einen solchen Freund in MAW gefunden hat.}
      - Jen T. (Entwicklerin von MAW)
\item \enquote{Ich hasse Sand. 
      Er ist kratzig und rau und unangenehm, aber kein Problem mehr dank MAW!}
      - Anakin S. (Sandsensitiver Sithlord)
\item \enquote{MAW war der einzige, der meine Pfannkuchen gegessen hat.
      Das rechne ich ihm hoch an.}
      - Pash W. (Bruchpilot und Hobbykoch)
\item \enquote{Der ehrbare Kaltho spricht eine Kaufempfehlung aus.
      MAW ist viel besser darin, den Fußboden mit dem Mund zu saugen als dumme Twi'lek-Sklaven es sind.}
      - Kaltho (Ehrbarer Hutte)
\item \enquote{Hat jemand meine Katze gesehen?}
      - Lyv (Ehemalige Haustierbesitzerin)
\end{itemize}
\Verfasser[Star Wars - Am Rande des Imperiums]{Franca}

\section{Todesanzeigen}
\textsc{Rest in Pieces}
\begin{center}Unser geliebter Staubsaugerdroide Maw.\end{center} 
Eine schwere Kiste fiel auf ihn herab und zertrümmerte ihn bei der Pflichterfüllung.
Wir vermissen dich, Maw, und dementieren außerdem, dass irgendetwas anderes als diese Kiste für dein Ende verantwortlich war. 
\Verfasser[Star Wars - Am Rande des Imperiums]{Franca}

\section{Werbung}
\textsc{Coming Soon! MAW 2.0!}
Dein alter MAW liegt in Trümmern und du hast irgendwie das Gefühl, dass deine Crew dran schuld ist? 
Dann rüste jetzt auf! MAW 2.0 verfügt über einen fortschrittlichen Schildgenerator und einen Korpus aus mandalorianischem Beskar. Ausfahrbare Defensiv-Spikes dienen der Selbstverteidigung und verhindern, dass der emsige Staubsaugerdroide versehentlich von Fracht oder Crewmitgliedern zerschmettert wird. 
MAW 2.0 - jetzt vorbestellen!
\Verfasser[Star Wars - Am Rande des Imperiums]{Franca}


\begin{termine}
% Put dates here:
\item Monatstreffen: 16.03.22, 19 Uhr
\end{termine}
\impressum

\end{document}
%%%%%%%%%%%%%%%%%%%%%%%%%%%%%%%%% END DOCUMENT %%%%%%%%%%%%%%%%%%%%%%%%%%%%%%%%%
