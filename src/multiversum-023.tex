% !TeX TS-program = xelatex
% !TEX root = main.tex

% How to:
%


% Load class with all definitions. Do not remove this line.
% Options will be passed to Memoir
\documentclass[final]{multiversum}
%

%%%
% Set those variables

% Authors of the document.
% e.g. Max Mustermann, Erika Musterfrau
\multiauthor{Franca, Hanna}

% Date of release.
% e.g. 31.12.2074
\multidate{Februar 2022}

% Number of release, no leading zeros.
% e.g. 15
\multiausgabe{23}

% Losung
% e.g. Die Kuh lief um den Teig.
\multilosung{Die sind fanatisch, aber nett.}


% Logo
% Use a different logo. Defaults to Ueberschrift.svg
%\multilogo{Ueberschrift_xmas}

%
%%%

%%%%%%%%%%%%%%%%%%%%%%%%%%%%%%%%% DOCUMENT %%%%%%%%%%%%%%%%%%%%%%%%%%%%%%%%%
\begin{document}

\makemultititle
%

% PUT BODY HERE
\section{Was bisher geschah...}

% \subsection{Workshop: ABC}
% Lorem Ipsum
% \verfasser{Autor1}

\subsection{In der Redaktion}
*Haaaaatschi!*

\enquote{Ganz schön staubig, wenn das Multiversum sich wieder zusammmensetzt!}

Mit dem anderen Praktikanten im Schlepptau stolpere ich durch das violette Portal zurück in die Redaktion.
Fast rutsche ich auf einem herumliegendem Stück Papier aus.
Der Sitzungsraum sieht immer noch genauso verwüstet aus wie ich ihn verlassen habe.
$\dots$Obwohl.
War es hier auch schon so staubig, als ich heute Morgen in die Redaktion gekommen bin? 
Man kann genau sehen, dass wohl drei oder vier von den Plätzen an dem großen Tisch genutzt werden, der Rest liegt in einer dicken Staubschicht verborgen.
Es führt eine nahezu staubfreie Spur vom Tisch zur Tür.
Es ist auch nicht mehr vollkommen still, ich kann einen Scanner und das Geraschel von Papier hören.

Aus einem Urinstinkt heraus verwische ich meine Spuren im Staub und wirble dabei noch mehr davon auf. 
Das Portal hat einen ellipsenförmigen Abdruck hinterlassen und wir haben schon einen kleinen Bereich zertrampelt.

*Hatschi!*

Der Praktikant muss auch husten.
\enquote{Jetzt kann ich dir zeigen, wie es hier zugeht.
Normalerweise sitzen hier $\dots$ 
Oh, das ist ja merkwürdig $\dots$ 
Na, auf jeden Fall sitzen hier die Redaktionsmitglieder und beraten über die Artikel für die neuen Ausgaben.
Es ist wichtig, dass du ihnen dazu Kaffee anbietest, sonst werden sie grantig.
Sag mal, war es schon so schmutzig, als du durch das Portal gekommen bist?}

Nun, da ich mich umblicke, sehe ich immer noch den aufgebockten Ring, die Papiere und die Tintenflecken, aber einige der Papiere sind vergilbt und in den Raumecken haben sich Spinnen eingenistet.

\enquote{Es sah so verwüstet aus, ja, aber ich habe das sauberer in Erinnerung}, sage ich stirnrunzelnd.

Ich folge dem Pfad zur Tür, den Praktikanten auf den Fersen.
Der Flur ist ruhig, aber wir können den Geräuschen in den Bürobereich folgen.
Ein paar Gestalten hängen mehr an den Schreibtischen als dass sie sitzen.
Sie haben alle lange Haare und tragen Kapuzenpullis.
Als wir den Raum betreten, drehen sich all ihre Köpfe langsam asynchron zu uns.
Niemand sagt etwas, bis alle Köpfe zu uns gedreht sind und wir zucken kollektiv zusammen, als der Kopierer aufheult und zu einem neuen Scan ansetzt.

\enquote{Kaf-fee?}, fragt der mir am nächsten sitzende Mitarbeiter.
Er sieht wirklich müde aus, seine Augenringe zieren sich über den Bereich zwischen seinen Augenbrauen und seinen Mundwinkeln.
Alle Köpfe drehen sich ruckartig zu ihm um, dann langsam zu uns zurück.

\enquote{Kaa-ffee?}, fragt die Redakteurin zu meiner rechten.
Sie richtet sich langsam aus ihrer hängenden Haltung und versucht unsicher, sich aus dem Stuhl zu erheben.
Auch der Redakteur daneben rappelt sich langsam auf.

Ich spüre, wie der Praktikant mich sacht am Arm zieht.
Wir machen vorsichtig einen Schritt rückwärts, in die Richtung der Teeküche.

\enquote{Kaaaffeeee}, kommt es nun vielstimmig aus allen Richtungen und die ersten Redakteure machen unsichere Schritte in unsere Richtung.
Schritt hinter Schritt, immer schneller, kommen sie auf uns zu, und Schritt für Schritt gehen wir rückwärts auf die Teeküche zu, bis wir die Tür in unserem Rücken haben und uns durch den Spalt quetschen können.
Der Praktikant wirft die Tür zu und ich verbarrikardiere sie schnell mit einem Stuhl.

\enquote{Das meine ich mit wichtig, wenn ich von Kaffee spreche}, erklärt er, während er mir den Wassertank der Kaffeemaschine zuwirft.
In schnellem Teamwork bestücken wir die Maschine und bald hört man neben den Rufen und dem Klopfen aus dem Flur auch das Fauchen von heißem Wasser.
Das Klopfen wird lauter, zumal sich das Aroma von frischem Kaffee breit macht.

\enquote{Wie viele waren das nochmal?}, frage ich und versuche, die richtige Anzahl nicht einheitlicher Tassen herauszusuchen.
\enquote{Egal, mach voll!}
Ich bekomme ein Tablett zugeworfen.

Als ich ein Tablett mit dampfenden Tassen, Milch und Zucker in der Hand habe, stellt sich der Praktikant seitlich neben die Tür.
\enquote{Ok, eins, zwei, $\dots$ drei!}
Schwungvoll öffnet er die Tür und die Redakteure stolpern hinein.
\enquote{Kaaaffeee!}
Müde und schleppend, aber nun deutlich schneller, nehmen sich die Redakteure Kaffee.

Eine kurze Weile später sind alle versorgt und die Rufe wandeln sich zu artikulierterem, zum Beispiel \enquote{Milch? Gibt es auch noch Hafer?}, um.
Während wir die Maschine für eine zweite Runde bestücken, kommt der Chef herüber.
\enquote{Hört mal, ihr beiden seid zu spät. Oder besser gesagt}, er wendet sich zu mir, \enquote{du bist zu spät.
Ihn hatte ich gar nicht mehr erwartet.}
Er deutet auf den Praktikanten.
\enquote{Praktikanten sind bei uns sehr wichtig und geschätzt.
Wir brauchen eure Fähigkeiten und frischen Ideen. 
Und den Kaffee.}
Er nimmt noch einen Schluck.
\enquote{Ohne den Kaffee sind wir ganz schön in Rückstand geraten. 
Uns fehlen in den letzten Monaten ein, ähm, paar Ausgaben.}
Wohlwollend nickt er uns mit seiner Tasse zu.
\enquote{Das bekommen wir nicht mehr aufgeholt, aber ab jetzt können wir wieder richtig arbeiten.
Da hilft nur eins. 
Kommt mal mit.}

Der Chef führt uns zu einer Tür ganz hinten im Flur. 
Das Schild besagt \enquote{Abstellkammer}.
Er holt einen Schlüssel heraus, dreht ihn ihm Schloss, klopft drei mal an bestimmten Stellen gegen die Tür und flüstert eine Passphrase ins Schlüsselloch.
Dann macht es \enquote{klick} und die Tür öffnet sich knarzend.
Im Licht des immer größer werdenen Türspalts wird langsam ein großer, roter Knopf sichtbar.
Darunter ist eine Innschrift angebracht.
Sie lautet: \enquote{RESET}.

\enquote{Damit kann man das Multiversum zurücksetzen.
Und natürlich damit auch das Redaktionsdatum}, erklärt der Chef und nimmt noch einen Schluck Kaffee.

Der Praktikant und ich schauen uns an.
In diesem Moment weiß ich, dass wir beide dasselbe denken:
\enquote{Echt jetzt?}

\verfasser{Die Praktikantin}

% \section{An einem anderen Ort}

% \subsection{Titel 1}
% Lorem Ipsum
% \verfasser{Autor2}

% \subsection{Titel 2}
% Lorem Ipsum, Text endet an Zeilenende.
% \Verfasser[System]{Autor2}

\section{Werbung}

% \subsection{Der Staubsaugerdroide, den Sie suchen!}

Die Hygieneroutinen des Staubsaugerdroiden MAW lassen die Prozessoren jedes Medidroiden höher schlagen!
Lästiger Asteroidenstaub im Frachtraum?
Verkohlte Asche von ungebetenen Besuchern auf der Laderampe?
Katzenhaare im Schmugglerversteck? Wookieepelz im Crewquartier?\\
\begin{center}\textsc{Kein Problem für MAW mit seiner antiallergenen Feinstaubfiltertechnologie nach modernsten Coruscanter Standards!}\\\end{center}
Bacta-Spritzer im Medizinraum?
Verschütteter Ryloth-Schnaps in der Küche?
Blutspritzer von Freund und Feind im Cockpit?\\
\begin{center}\textsc{Die auf Mon Cala entwickelte Flüssigsaugfunktion wird auch damit fertig!}\\\end{center}

Ryll Spice im Frachtraum?
Überreste explodierter Rodianer?
Die kulinarischen Experimente des Piloten?
\begin{center}\textsc{Innerhalb von Minuten kann MAW jedes Beweisstück verschwinden lassen!}\\\end{center}
Ausgerüstet mit einem neuartigen Navigationssystem, einem Adapter für alle herkömmlichen Hyperraumklappen, langanhaltendem Akku, großer beutelfreier Saugkapazität und parkettschonenden Rollen ist \textbf{MAW} bereit, jedes versiffte Raumschiff wieder auf Vordermann zu bringen.
\begin{center}\textsc{Dieser kleine Staubsaugerdroide würde selbst Tatooine im Handumdrehen sauber bekommen!}\\\end{center}
Bestellen Sie noch heute den Staubsaugerdroiden MAW für nur *2000 Credits* und erhalten Sie wahlweise einen Blaster oder einen modischen Hut zu Ihrer Bestellung dazu. 
Bestellungen holografisch oder per Textnachricht an Ra'an Thashin.\\
\begin{center}\textsc{*Kundenstimmen*}\\\end{center}
\begin{itemize}
\item \enquote{MAW versteht sich ausgezeichnet mit MINI, unserem Flammenwerfer-Droideka.
      MAW saugt ohne Murren die Asche hinter MINI auf und die zwei sind direkt unzertrennlich geworden. 
      Es ist schön, dass unser MINI einen solchen Freund in MAW gefunden hat.}
      - Jen T. (Entwicklerin von MAW)
\item \enquote{Ich hasse Sand. 
      Er ist kratzig und rau und unangenehm, aber kein Problem mehr dank MAW!}
      - Anakin S. (Sandsensitiver Sithlord)
\item \enquote{MAW war der einzige, der meine Pfannkuchen gegessen hat.

      Das rechne ich ihm hoch an.}
      - Pash W. (Bruchpilot und Hobbykoch)
\item \enquote{Der ehrbare Kaltho spricht eine Kaufempfehlung aus.
      MAW ist viel besser darin, den Fußboden mit dem Mund zu saugen als dumme Twi'lek-Sklaven es sind.}
      - Kaltho (Ehrbarer Hutte)
\item \enquote{Hat jemand meine Katze gesehen?}
      - Lyv (Ehemalige Haustierbesitzerin)
\end{itemize}
\verfasser[Star Wars - Am Rande des Imperiums]{Franca}

\textsc{Rest in Pieces}
\begin{center}Unser geliebter Staubsaugerdroide Maw.\end{center} Eine schwere Kiste fiel auf ihn herab und zertrümmerte ihn bei der Pflichterfüllung.
Wir vermissen dich, Maw, und dementieren außerdem, dass irgendetwas anderes als diese Kiste für dein Ende verantwortlich war. 

\textsc{Coming Soon! MAW 2.0!}
Dein alter MAW liegt in Trümmern und du hast irgendwie das Gefühl, dass deine Crew dran schuld ist? 
Dann rüste jetzt auf! MAW 2.0 verfügt über einen fortschrittlichen Schildgenerator und einen Korpus aus mandalorianischem Beskar. Ausfahrbare Defensiv-Spikes dienen der Selbstverteidigung und verhindern, dass der emsige Staubsaugerdroide versehentlich von Fracht oder Crewmitgliedern zerschmettert wird. 
MAW 2.0 - jetzt vorbestellen!
\verfasser[Star Wars - Am Rande des Imperiums]{Franca}

\begin{termine}
% Put dates here:
\item Termin: Termin: DD.MM.YY, hh Uhr
  \item Termin: DD.MM.YY - DD.MM.YY
\end{termine}
\impressum

\end{document}
%%%%%%%%%%%%%%%%%%%%%%%%%%%%%%%%% END DOCUMENT %%%%%%%%%%%%%%%%%%%%%%%%%%%%%%%%%
