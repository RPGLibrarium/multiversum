% !TeX TS-program = xelatex
% !TEX root = main.tex

% How to:
%


% Load class with all definitions. Do not remove this line.
% Options will be passed to Memoir
\documentclass[final]{multiversum}
%

%%%
% Set those variables

% Authors of the document.
% e.g. Max Mustermann, Erika Musterfrau
\multiauthor{Hanna, Yoann}

% Date of release.
% e.g. 31.12.2074
\multidate{01.01.2025}

% Number of release, no leading zeros.
% e.g. 15
\multiausgabe{30}

% Losung
% e.g. Die Kuh lief um den Teig.
\multilosung{Damit kommen wir zu den lustigen Sachen: Dienstverweigerung!}


% Logo
% Use a different logo. Defaults to Ueberschrift.svg
%\multilogo{Ueberschrift_xmas}

%
%%%

%%%%%%%%%%%%%%%%%%%%%%%%%%%%%%%%% DOCUMENT %%%%%%%%%%%%%%%%%%%%%%%%%%%%%%%%%
\begin{document}

\makemultititle
%

\subsection{Die Phileasson-Saga, ein Review}
\emph{Spoilerwarnung: Ein Teil des Inhalts wird hier
angedeutet, wenn auch nicht vollständig verraten.}

Die Phileasson-Saga gilt als eines der großen DSA-Abenteuer, die man einmal
gespielt haben sollte. Ursprünglich ist dieses Abenteuer von Bernhard Hennen
1990 -- 1991 als vierteiliges Abenteuer für DSA~2 erschienen (auch in der
Bibliothek des Librariums zu finden). Die Saga wurde sowohl 1999 für DSA~3
als auch 2009 für DSA~4.1 neu aufgelegt. Dieser Review bezieht sich auf die
Neuauflage für DSA~4.1.

Die Geschichte stellt sich als Seefahrtsabenteuer mit Umsegelung eines
Kontinents dar. In der Einleitung spielen wir ein großes Fest in Thorwal, der
Hauptstadt der den Wikingern nachempfundenen Thorwaler. In einer gemütlichen
Halla eskalieren die Feierlichkeiten zur Wintersonnenwende, als sich unter den
Versammelten Hetleuten Asleif \enquote{dem Foggwulf} Phileasson und Beorn
\enquote{dem Blender} eine Streiterei entwickelt. Beorn behauptet, einen anderen
Kontinent erreicht zu haben, und Phileasson -- der schon dort war -- bezichtigt
ihn ob seiner Beschreibungen der Lüge. Zur Schlichtung dieses Streites verfügt
die oberste Hetfrau eine Wettfahrt zwischen den beiden Kapitänen. Diese ist eine
Mischung aus den 12~Aufgaben des Herakles und \emph{In 80 Tagen um die Welt} und
soll dem Gewinner den Titel \enquote{König der Meere} einbringen. Die Kapitäne
dürfen dabei nicht ihre übliche Mannschaft mitnehmen, um ihre Führungskraft
demonstrieren zu müssen. Diese durchsichtige Ausrede dient dazu, die
Heldengruppe einzusammeln, die sofort bei Phileasson anheuert. Beorn nimmt nur
Thorwaler aus einer bestimmten Region mit und ist rassistisch, damit die~Helden
den~\enquote{richtigen} Kapitän auswählen. Jede Gruppe bekommt eine Geweihte
mit, die als Schiedsrichterin dient und die jeweils nächste Aufgabe verkünden
kann.

Im Gegensatz zur ursprünglichen Ausgabe, in der die 12 Aufgaben auf vier dünne
Hefte aufgeteilt waren, erschlägt den Spielleiter bei der Ausgabe für DSA~4.1
ein fast 300~Seiten langes Abenteuer. Die Kampagne ist auf zwei~Jahre Spielzeit
ausgelegt. Im Abenteuerband finden sich einige Ausgestaltungen der Welt als
Meisterinformationen, die vor allem die Entstehungsgeschichte der Handlungsorte
behandeln. Für den Meister macht das die Orte lebendiger. Leider haben die
Helden keine Chance, auch nur einen Bruchteil dieser Informationen mitzubekommen
und das macht es den Spielern schwer, die Hintergründe des Abenteuers zu
erfahren. 
% Viele
% 2000 Jahre alte Bauwerke und Relikte sind zu finden, die ohne Hintergrundwissen
% kaum zu erfassen sind.

Die 12~Aufgaben können nicht nach und nach vom Spielleiter aufgearbeitet werden,
da es in dieser Ausgabe viele Querverweise gibt. Einige Aspekte, die in den
ersten Aufgaben nur kurz erwähnt werden, werden am Ende plötzlich wichtige,
plotrelevante Elemente. Einige als wichtig dargestellte Elemente kommen dafür
nie wieder vor. Als Meister muss man also das gesamte Abenteuer durcharbeiten
und sich die Kleinigkeiten merken, die anfangs vorkommen müssen, damit das
Abenteuer später noch Sinn ergibt. Diese Details müssen natürlich für die
nächsten zwei~Jahre im Gedächtnis bleiben. Tatsächlich brauchte meine Runde
fünf~Jahre, denn das Abenteuer ist nicht für umfassendes Charakterplay
ausgelegt. Hier ist besser aufgehoben, wer questlastige Abenteuer mag.

Dieses Abenteuer ist ein gefundenes Fressen für die Freunde des gepflegten
Railroadings. Durch die Mannschaft, mit der man unterwegs ist, kann man sich als
Mitläufer verdingen und Phileasson seine Führung ausleben lassen. Die Spielrunde
muss also eine Balance finden, wie die Helden sich einbringen können, die
Mannschaft und insbesondere Kapitän Phileasson aber nicht vollständig irrelevant
werden. Unsere Mutmaßung ist, dass Bernhard Hennen seinerzeit dieses
Abenteuer mit seiner eigenen Runde gespielt und dabei die Lösungen seiner
Spieler eingebaut hat.

Somit gibt es beispielsweise einen aufwendigen Seekampf, der laut Abenteuer um
keinen Preis gewonnen werden darf, um eine weitere Szene zu ermöglichen, in der
man ein benötigtes Monsterteil von einem toten oder verletzten Monster entnehmen
kann. Diese zweite Szene gibt nur begrenzten Mehrwert, etabliert aber einen Ort,
der in der Historie der DSA-Welt eine große Relevanz fürs Worldbuilding gewonnen
hat. Daher wäre es schade diesen Ort ganz wegzulassen. Unsere Mutmaßung:
Hennens Runde verlor leider den ersten Kampf und brauchte eine
alternative Lösung.

Im Abenteuerband gibt es eine Aufzählung davon, welche Arten von Charakteren für
die Reise nicht geeignet sind. Darin fehlt leider ein Hinweis zur Gesinnung der
Runde. Beorn ist der böse, brutale Kapitän, der in seiner Grausamkeit seine
verquere Moral durch das Blenden von Sklavenhändlern auslebt. Phileasson
hingegen ist ein viel netterer Pirat, der den Helden dadurch direkt
sympathischer ist, denn er würde Sklavenhändler nur töten, weil sie böööse sind.
Hm. Tatsächlich geht das Abenteuer von einer moralisch flexiblen Gruppe aus, die
sich nicht unbedingt an Recht und Ordnung hält. Zugegeben, wir haben hier ein
Abenteuer mit Piraten, doch eine kleine Warnung wäre nett gewesen. So gibt es
beispielsweise einen Konflikt, wenn die Helden ein Schiff suchen und nur hören,
dass ein gefährlicher gesuchter Schmuggler sich herumtreibt. Das Abenteuer sieht
vor, dass die Gruppe sich mit diesem Schmuggler sofort bedingungslos anfreundet.

Außerdem geht Hennen von einer Gruppe aus, die am Ende jeden Abschnitts einen
Bosskampf erwartet. Das Abenteuer eignet sich daher nicht für Gurkentruppen.

Soweit zu den Aspekten des Abenteuers, die auf See ablaufen. Für ein
Seefahrtsabenteuer ist die Mannschaft sehr wenig auf See unterwegs. Beim ersten
Teil des Abenteuers segelt Phileasson einmal um eine Bucht, dann geht es erst
einmal eine Weile lang auf Eisgleitern weiter, und man kreuzt die Bucht noch
einmal. Der dritte Teil des Abenteuers beginnt auf Land und ab da muss man
einmal quer über den Kontinent wandern, ohne Schiff, zu Fuß von der Westküste
zur Ostküste. Von dort aus geht es nach Süden und es gibt auch kein neues Schiff
-- zwar kann man Teilabschnitte auf Handelskoggen zurücklegen, doch es geht für
die eigentlichen die Aufgaben immer wieder landeinwärts. Weit genug im Süden
angekommen, in der Mitte des Kontinents, geht es einmal mit einer Karawane durch
eine Wüste zurück zur Westküste. Nun bekommen wir wieder ein Schiff und segeln
tatsächlich von dort aus weiter. Sobald man die \enquote{Runde um den Kontinent}
beendet hat und zurück in Thorwal ist, hat man\dots{} ganze 9 von 12 Aufgaben
gemeistert. Man ist zurück, feiert ein großes Fest und darf dann wieder
losziehen. Ob irgendeine Runde das nach einem so langwierigen Abenteuer noch
macht, ist mir unklar.

Da meine eigene Runde das Railroading satt war, haben wir die letzten paar
Aufgaben auf einen Haufen geschmissen, an zufälligen, ungefähr passenden
Momenten wurden die nächsten Hinweise hinein geschmissen. Durch die wochenlangen
Reisezeiten hat man sehr viel Downtime, sodass man nach den echten DSA-Regeln
leveln kann. Leider ist das sehr langwierig und meine Gruppe bekam eine in DSA
nicht vorgesehene Schnellreiseoption. (Portalmagie existiert in der DSA-Lore,
aber ist in den Regeln absolut nicht vorgesehen.)

Reden wir über Antagonisten. Das ist natürlich Beorn, der Konkurrent! Zumindest
denken die Spieler das. Der Meister hat da noch Tonnen an weiteren
Informationen, wer im Hintergrund noch so alles die~Fäden zieht, und da gibt es
eine ganze Menge Einflüsse. Leider können die Helden das realistisch gesehen
niemals herausfinden und so bekommen sie nichts von den enormen Mächten mit, mit
denen sie spielen. Es gibt nur obskure Andeutungen, die die Spieler eher
verwirren und auf falsche Fährten führen. Der Meister kann dies höchstens
nutzen, um die Spieler mit undurchsichtigen Manövern aus Notsituationen zu
retten, Deus ex machina!

Ein letzter Punkt: Die Saga hat sehr viele Anspielungen aus zeitgemäßer
Literatur, Weltgeschehen und Popkultur. Aus heutiger Sicht gibt es daher eine
ganze Menge veralteter Referenzen. Als Meister kann man sich entscheiden, wie
viele man davon einfließen lassen will, allerdings sind einige fester
Bestandteil der Erzählung. Das war zur Zeit der Entstehung wahrscheinlich
amüsanter.

Alles in allem ist die Phileasson-Saga ein Relikt mit zu viel Inhalt. Dieses
Abenteuer hat die aventurische Historie sehr geprägt und ist in dieser Hinsicht
durchaus spannend. Dadurch werden manche Teile zur Museumstour. (Hinweis im
Abenteuer: \enquote{Nicht zur Museumstour verkommen lassen.} Aber womit füllen,
ohne das Railroading zu zerstören?) Trotzdem ist die Geschichte immer noch
interessant.

Die Phileasson-Saga wird häufig als tolles Abenteuer für DSA genannt,
insbesondere auch für Einsteiger. Sie ist definitiv nicht für einsteigende
Spielleiter geeignet. Da es inzwischen auch eine Romanreihe der
Phileasson-Saga gibt, kann man stattdessen auch diese lesen. Meine Empfehlung
ist, ein moderneres Abenteuer (und wahrscheinlich auch ein moderneres System mit
einer moderneren Welt) zu wählen.
\Verfasser[DSA 4.1]{Hanna} 

\section{Werbung}
Jetzt \textsc{neu} in Ihrer Stadt: Raclette \textsc{Ex Magronax}! Stillen Sie Ihren
Heißhunger auf duftenden und geschmolzenen Käse bei einem spontanen Raclette.
Wir bieten ein Sortiment aus vielen exklusiven Käsesorten.

Alle unsere Käseräder sind in flexiblen Größen zu haben. Greifen Sie zur Portion
\emph{Little Dwarf} für den kleinen Hunger zwischendurch oder wählen Sie die Option
\emph{Infinite Cheese} im Falle einer unvorbereiteten Sonnenwende für bis zu 60
Personen. Egal wie Sie sich entscheiden, wir liefern Ihre Bestellung garantiert
innerhalb von 60 Minuten frei Haus!

Im Mittagsmenü erhalten Sie zu unseren Käserädern zusätzlich ein eisgekühltes Getränk
gratis. Zur Wahl stehen Ihnen unsere allzeit beliebte Bestseller-Limonade
Yngwar-Orange oder eine Flasche alkoholfreier Rottrunk.

Eines ist gewiss: Mit dem Raclette von \textsc{Ex Magronax} bekommen Sie jeden
Heißhunger und Durst gestillt. Bestellen Sie noch heute und sparen Sie auf Ihre
erste Bestellung 15 Silbertaler mit dem Code \emph{TROLLZACKER15}!
\Verfasser[DSA4.1]{Yoann}
\begin{termine}
  \item Freizeit: 04.01 -- 08.01.25, Freusburg
  \item Monatstreffen: 16.01.25, 19 Uhr, Papillon
\end{termine}
\impressum

\end{document}
%%%%%%%%%%%%%%%%%%%%%%%%%%%%%%%%% END DOCUMENT %%%%%%%%%%%%%%%%%%%%%%%%%%%%%%%%%
