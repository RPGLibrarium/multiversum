% !TeX TS-program = xelatex
% !TEX root = main.tex

% How to:
%


% Load class with all definitions. Do not remove this line.
% Options will be passed to Memoir
\documentclass[final]{multiversum}
%

%%%
% Set those variables

% Authors of the document.
% e.g. Max Mustermann, Erika Musterfrau
\multiauthor{Hanna}

% Date of release.
% e.g. 31.12.2074
\multidate{}

% Number of release, no leading zeros.
% e.g. 15
\multiausgabe{30}

% Losung
% e.g. Die Kuh lief um den Teig.
\multilosung{Platzhalter}


% Logo
% Use a different logo. Defaults to Ueberschrift.svg
%\multilogo{Ueberschrift_xmas}

%
%%%

%%%%%%%%%%%%%%%%%%%%%%%%%%%%%%%%% DOCUMENT %%%%%%%%%%%%%%%%%%%%%%%%%%%%%%%%%
\begin{document}

\makemultititle
%

\subsection{Die Phileasson-Saga, Überarbeitete Version für DSA 4.1, ein Review}
Die Phileasson-Saga gilt als eines der großen DSA-Abenteuer, die man einmal
gespielt haben sollte. Ursprünglich ist dieses Abenteuer von Bernhard Hennen
1990-1991 als vierteiliges Abenteuer für DSA 2 erschienen (das auch in der
Bibliothek des Librariums zu finden ist). Die Saga wurde sowohl 1999 für DSA 3
als auch 2009 für DSA 4.1 neu aufgelegt. Dieser Review bezieht sich auf die
Neuauflage für DSA 4.1. Spoilerwarnung: Ein Teil des Inhalts wird hier
angedeutet, wenn auch nicht vollständig gespoilert.

Diese Saga stellt sich als Seefahrtsabenteuer mit Umsegelung eines Kontinents
dar. In der Einleitung spielen wir ein großes Fest in Thorwal, der Hauptstadt
der den Wikingern nachempfundenen Thorwaler. In einer gemütlichen Halla
eskalieren die Feierlichkeiten zur Wintersonnenwende, als sich unter den
Versammelten Hetleuten Asleif (dem Foggwulf) Phileasson und Beorn dem Blender
eine Streiterei entwickelt. Beorn behauptet, einen anderen Kontinent erreicht zu
haben, und Phileasson, der schon dort war, bezichtigt ihn ob seiner
Beschreibungen als Lügner, da Phileasson selbst diesen Kontinent schon erforscht
hat. Zur Schlichtung dieses Streites verfügt die oberste Hetfrau eine Wettfahrt
zwischen den beiden Kapitänen, die aus 12 Aufgaben bestehen soll, 80 Wochen
dauert und dem Gewinner den Titel “König der Meere” einbringt. Die Kapitäne
dürfen dabei nicht ihre übliche Mannschaft mitnehmen, um ihre Führungskraft
einbringen zu müssen. Diese durchsichtige Ausrede dient dazu, die Heldengruppe
einzusammeln, die natürlich sofort bei Phileasson anheuert. Beorn nimmt nur
Thorwaler aus einer bestimmten Region mit und ist rassistisch, damit es da keine
Probleme gibt. Jede Gruppe bekommt eine Geweihte mit, die als Schiedsrichterin
dient und durch Göttervisionen die jeweils nächste Aufgabe verkünden kann.

Das Abenteuer ist in 12 Teile aufgeteilt. Im Gegensatz zur ursprünglichen
Ausgabe, die in vier Hefte aufgeteilt war, erschlägt den Spielleiter bei der
Ausgabe für DSA 4.1 ein fast 300 Seiten langes Abenteuer. Die Kampagne ist auf 2
Jahre Spielzeit ausgelegt.

Leider können die 12 Aufgaben nicht nach und nach aufgearbeitet werden, da es
viele Querverweise gibt und einige Aspekte, die in den ersten Aufgaben nur kurz
erwähnt werden, am Ende plötzlich wichtige plotrelevante Elemente werden. Einige
als wichtig dargestellte Elemente kommen dafür nie wieder vor. Als Meister muss
man also dieses gesamte Abenteuer durcharbeiten und sich die Kleinigkeiten
merken, die anfangs vorkommen müssen, damit das Abenteuer später noch Sinn
ergibt. Und diese Details müssen natürlich die nächsten 2 Jahre im Gedächtnis
verbleiben. Realistisch gesehen brauchte meine Runde 5 Jahre, das Abenteuer ist
nicht auf umfassendes Charakterplay ausgelegt. Wer questlastige Abenteuer mag,
ist hier besser aufgehoben.

Auf den 300 Seiten des Abenteuers finden sich einige Abschnitte an
Meisterinformationen. Dabei sind ganz viele Details, wie die besuchten Orte
entstanden sind und bei der Ausgestaltung helfen. Leider haben die Helden keine
Chance, auch nur einen Bruchteil dieser Informationen mitzubekommen und das
macht es den Spielern schwer, die Hintergründe des Abenteuers zu erfahren. Viele
2000 Jahre alte Bauwerke und Relikte sind zu finden, die ohne Hintergrundwissen
kaum zu erfassen sind.

Dieses Abenteuer ist ein gefundenes Fressen für die Freunde des gepflegten
Railroadings. Durch die Mannschaft, mit der man unterwegs ist, kann man sich als
Mitläufer verdingen und Phileasson seine Führung ausleben lassen. Die Spielrunde
muss also eine Balance finden, wie die Helden sich einbringen können, die
Mannschaft und insbesondere Kapitän Phileasson aber nicht vollständig irrelevant
werden. Eine plausible Mutmaßung ist, dass Bernhard Hennen seinerzeit dieses
Abenteuer mit seiner eigenen Runde gespielt hat und dabei die Lösungen seiner
Runde eingebaut hat.

Somit gibt es beispielsweise einen aufwendigen Seekampf, der laut Abenteuer um
keinen Preis gewonnen werden darf, um eine weitere Szene zu ermöglichen, wo man
das gesuchte Monsterteil von einem toten oder verletzten Monster entnehmen kann.

Diese zweite Szene gibt nur begrenzten Mehrwert. Leider etabliert diese Szene
einen Ort, der in der Historie der DSA-Welt eine große Relevanz fürs
Worldbuilding gewonnen hat, daher wäre es schade diese Szene wegzulassen. Unsere
Mutmaßung: Hennens Runde hat den Kampf leider verloren und brauchte eine
Alternative.

Zudem gibt es im Abenteuer eine Aufzählung davon, welche Arten von Charakteren
nicht geeignet sind. Darin fehlt leider ein Hinweis zur Gesinnung der Runde.
Beorn ist der böse, brutale Kapitän, der in seiner Grausamkeit seine verquere
Moral durch das Blenden von Sklavenhändlern auslebt. Phileasson hingegen ist ein
viel netterer Pirat, der den Helden dadurch direkt sympathischer ist, denn er
würde Sklavenhändler nur töten, weil sie böööse sind. Hm.

Tatsächlich geht das Abenteuer von einer moralisch flexiblen Gruppe aus, die
sich nicht unbedingt an Recht und Ordnung hält. Zugegeben, wir haben hier ein
Abenteuer mit Wikingern, doch eine kleine Warnung wäre nett gewesen. Zumindest
ein Hinweis für den Spielleiter wäre gut. So gibt es beispielsweise einen
Konflikt, wenn die Helden ein Schiff suchen und nur hören, dass ein gefährlicher
gesuchter Schmuggler sich herumtreibt. Das Abenteuer sieht vor, dass die Gruppe
sich mit diesem Schmuggler sofort bedingungslos anfreundet. Außerdem geht Hennen
von einer Gruppe aus, die am Ende jeden Abschnitts einen Bosskampf erwartet. Das
Abenteuer eignet sich hier nicht für Gurkentruppen.

Soweit zu den Aspekten des Abenteuers, die auf See ablaufen. Für ein
Seefahrtsabenteuer ist die Mannschaft sehr wenig auf See unterwegs. Beim ersten
Teil des Abenteuers segelt Phileasson einmal um eine Bucht, dann geht es erst
einmal eine Weile lang auf Eisgleitern weiter, und man kreuzt die Bucht noch
einmal. Der dritte Teil des Abenteuers beginnt auf Land und ab da muss man
einmal quer über den Kontinent wandern, ohne Schiff, zu Fuß von der Westküste
zur Ostküste. Von da aus geht es nach Süden und es gibt auch kein neues Schiff -
zwar kann man Teilabschnitte auf Handelskoggen zurück legen, doch es geht für
eigentlichen die Aufgaben immer wieder landeinwärts. Weit genug im Süden
angekommen, in der Mitte des Kontinents, geht es einmal in einer Karawane durch
eine Wüste zurück zur Westküste. Nun bekommen wir wieder ein Schiff und segeln
tatsächlich von da an weiter. Sobald man die Runde um den Kontinent beendet hat
und zurück in Thorwal ist, hat man… ganze 9 von 12 Aufgaben gemeistert. Man ist
zurück, feiert ein großes Fest und darf dann wieder los ziehen. Ob irgendeine
Runde das nach einem so langwierigen Abenteuer noch macht, ist mir unklar. Da
meine eigene Runde das Railroading satt war, haben wir die letzten paar Aufgaben
auf einen Haufen geschmissen, an zufälligen ungefähr passenden Momenten wurden
die nächsten Aufgaben hinein geschmissen. Und durch die wochenlangen Reisezeiten
hat man sehr viel Downtime, sodass man nach den echten DSA-Regeln leveln kann.
Leider ist das sehr langwierig und meine Gruppe bekam eine in DSA nicht
vorgesehene Schnellreiseoption (Portalmagie existiert in der DSA-Lore, aber ist
in den Regeln absolut nicht vorgesehen).

Reden wir über Antagonisten. Das ist natürlich Beorn, der Konkurrent! Zumindest
denken die Spieler das. Der Meister hat da noch Tonnen an weiteren
Informationen, wer im Hintergrund so alles noch Fäden zieht, und da gibt es eine
ganze Menge Einflüsse. Leider können die Helden das realistisch gesehen niemals
herausfinden und so bekommen sie nichts von den enormen Mächten mit, mit denen
sie spielen. Es gibt nur obskure Andeutungen, die die Spieler ehre verwirren und
auf falsche Fährten ansetzen. Der Meister kann das höchstens nutzen, um die
Spieler mit undurchsichtigen Manövern zu retten, Deus ex machina!

Ein letzter Punkt, die Saga hat sehr viele Anspielungen aus Literatur,
Weltgeschehen und Popkultur. Aus heutiger Sicht bekommen wir in der Saga eine
ganze Menge veralteter Referenzen. Als Meister kann man sich entscheiden, wie
viele man davon einfließen lassen will, allerdings sind diese häufig fester
Bestandteil. Das war zur Entstehung wahrscheinlich deutlich witziger.

Alles in allem ist die Phileasson-Saga ein Relikt mit viel Inhalt, dieses
Abenteuer hat die aventurische Historie sehr geprägt und ist in dieser Hinsicht
durchaus spannend. Dadurch werden manche Teile zur Museumstour (Hinweis im
Abenteuer: “Nicht zur Museumstour verkommen lassen.” Aber womit füllen, ohne das
Railroading zu zerstören?) Trotzdem ist die Geschichte immer noch interessant.

Die Phileasson-Saga wird häufig als tolles Abenteuer für DSA genannt,
insbesondere auch für Einsteiger. Sie ist definitiv nicht für einsteigende
Spielleiter geeignet. Da es dazu inzwischen auch eine Romanreihe der
Phileasson-Saga gibt, kann man stattdessen auch diese lesen. Meine Empfehlung
ist, ein moderneres Abenteuer (und wahrscheinlich auch ein moderneres System mit
einer weniger rassistisch erstellten Welt) zu wählen.
\verfasser[DSA 4.1]{Hanna} 

\section{Werbung}
Jetzt \textsc{neu} in Ihrer Stadt: Raclette \textsc{Ex Magronax}! Stillen Sie Ihren
Heißhunger auf duftenden und geschmolzenen Käse bei einem spontanen Raclette.
Wir bieten ein Sortiment aus vielen exklusiven Käsesorten.
Alle unsere Käseräder sind in flexiblen Größen zu haben. Greifen Sie zur Portion
`Little Dwarf` für den kleinen Hunger zwischendurch oder wählen Sie die Option
`Infinite Cheese` im Falle einer unvorbereiteten Sonnenwende für bis zu 60
Personen. Egal wie Sie sich entscheiden, wir liefern Ihre Bestellung garantiert
innerhalb von 60 Minuten frei Haus!

Im Mittagsmenü erhalten Sie zu unseren Käserädern zusätzlich ein eisgekühltes Getränk
gratis. Zur Wahl stehen Ihnen unsere allzeit beliebte Bestseller-Limonade
Yngwar-Orange oder eine Flasche alkoholfreier Rottrunk.

Eines ist gewiss: Mit dem Raclette von \textsc{Ex Magronax} bekommen Sie jeden
Heißhunger und Durst gestillt. Bestellen Sie noch heute und sparen Sie mit dem Code 
\emph{TROLLZACKER15} 15 Silbertaler auf Ihre erste Bestellung!
\Verfasser[DSA4.1]{Yoann}
\begin{termine}
  \item Freizeit: 04.01 -- 08.01.25, Freusburg
  \item Monatstreffen: 16.01.25, 19 Uhr, Papillon
\end{termine}
\impressum

\end{document}
%%%%%%%%%%%%%%%%%%%%%%%%%%%%%%%%% END DOCUMENT %%%%%%%%%%%%%%%%%%%%%%%%%%%%%%%%%
