% !TeX TS-program = xelatex
% !TEX root = main.tex

% How to:
%


% Include preable with all definitions. Do not remove this line.
%%%
% Document Class: Memoir
% http://texdoc.net/texmf-dist/doc/latex/memoir/memman.pdf
\documentclass[a4paper,10pt,oneside,twocolumn,final,article]{memoir}
%
%%%

%%%
% variables
\makeatletter

\newcommand\multiauthor[1]{\renewcommand\@multiauthor{#1}}
\newcommand\@multiauthor{\@latex@error{No \noexpand\multiauthor given}\@ehc}

\newcommand\multidate[1]{\renewcommand\@multidate{#1}}
\newcommand\@multidate{\@latex@error{No \noexpand\multidate given}\@ehc}

\newcommand\multiausgabe[1]{\renewcommand\@multiausgabe{#1}}
\newcommand\@multiausgabe{\@latex@error{No \noexpand\multiausgabe given}\@ehc}

\newcommand\multilosung[1]{\renewcommand\@multilosung{\expandafter\MakeUppercase\expandafter{#1}}}
\newcommand\@multilosung{\@latex@error{No \noexpand\multilosung given}\@ehc}


%%%

%%%
% Fonts and Typesetting

% Mathspec: used to change math font.
\usepackage{mathspec}

%IPA
\usepackage{tipa}
\renewcommand\textipa[1]{{\fontfamily{cmr}\tipaencoding #1}}

%
% We use the open Adobe Source font family
% https://github.com/adobe-fonts
%
% Roman (Serif) Font: Source Serif Pro Light/Semibold
\setprimaryfont[
UprightFont={* Light},
ItalicFont={* Light Italic},
BoldFont={* Semibold},
BoldItalicFont={* Semibold Italic},
Scale=MatchLowercase]{Source Serif Pro}
%
% Sans Font: Source Sans Pro Light/Semibold
\setallsansfonts[
UprightFont={* Light},
ItalicFont={* Light Italic},
BoldFont={* Semibold},
BoldItalicFont={* Semibold Italic},
Scale=MatchLowercase]{Source Sans Pro}
%
% Mono Font: Source Code Pro Light/Semibold (No Italics available)
\setallmonofonts[
UprightFont={* Light},
BoldFont={* Semibold},
Scale=MatchLowercase]
{Source Code Pro}
%
% Replace math greek fonts by Source Serif Pro (No Italics available -> Use fake italics)
\setmathfont(Greek)[
UprightFont={* Light},
BoldFont={* Semibold},
AutoFakeSlant=0.15
]
{Source Serif Pro}
%
%%%

%%% Colors!
\usepackage[table]{xcolor}
%%%

%%%
% Memoir Class Page formatting
%
% Compute a suitable line width.
\setlxvchars \setxlvchars
% type block size:
\settypeblocksize{700pt}{1.6\lxvchars}{*}
% type block at the center of the page:
\setlrmargins{*}{*}{1}
\setulmargins{*}{*}{1}
% separator between two columns, no rule:
\setcolsepandrule{0.5cm}{0pt}
% header and footer height:
\setheadfoot{2\onelineskip}{2\onelineskip}
% note margin width and separator:
\setmarginnotes{17pt}{3.5cm}{\onelineskip}
% save changes and fix the layout:
\checkandfixthelayout
% empty page layout (no header, no footer)
\pagestyle{empty}
%
%%%


%%%
% Balance last two columns
% Set "balance" variable to "true" in pandoc to balance the columns on the last
% page.
%$if(balance)$\usepackage{flushend}$endif$
%%%


%%%
% Language
\usepackage[english,ngerman]{babel}
\usepackage[german=quotes]{csquotes}
%
%%%


%%%
% Figures
\usepackage{graphicx}
%
%%%

%%%
% Date Format:
\usepackage[ngerman]{datetime}
\newdateformat{monthyear}{\monthname[\THEMONTH] \THEYEAR}
%
%%%


%%%
% Licensing
\usepackage[
	type={CC},
	modifier={by-nc-sa},
	version={4.0},
]{doclicense}
\usepackage{rotating}
\usepackage{wrapfig,ragged2e}
%%%


%%%
% Meta data
\author{\@multiauthor}
\date{\@multidate}
\title{Multiversum Ausgabe Nr. \@multiausgabe}
%
%%%


%%%
% Cross-References, Links and PDF-Metadata
\usepackage{hyperref}
% Moved to makemultititle
% \hypersetup{
%	pdftitle    = {Multiversum Ausgabe Nr. \@multiausgabe},
%	pdfsubject  = {RPG Librarium Aachen e.V.},
%	pdfauthor   = {\@multiauthor},
%	pdfcreator  = {XeLaTex},
%	hidelinks,
%}
\urlstyle{sf}
%
%%%


%%%
% Divisions
% Section Style
\setbeforesecskip{-.5em}
\newcommand{\ruledsec}[1]{%
	\LARGE\scshape\centering #1 \rule{\linewidth}{0.4pt}\noindent}
\setsecheadstyle{\ruledsec}
\setaftersecskip{.5em}
% Subsection Style
\setsubsecheadstyle{\Large\normalfont\raggedright}
% Paragraph Style
\setparaheadstyle{\bfseries}
% Disable numbering
\setsecnumdepth{part}
% include subsections in PDF bookmarks
\setcounter{tocdepth}{2}
%
%%%


%%%
% Lists, Enumerations, Itemize
% tight spacing
\tightlists
%
%%%

%%%
% Cross out
\usepackage[normalem]{ulem}
%%%

%%%
% Macros
%

% Title
%
% Use this command to set a nice looking title.
% first argument: number, e.g. 15
% second argument: losung, e.g. Die Kuh lief um den Teich.
%
\newcommand{\makemultititle}{
\hypersetup{
  pdftitle    = {Multiversum Ausgabe Nr. \@multiausgabe},
  pdfsubject  = {RPG Librarium Aachen e.V.},
  pdfauthor   = {\@multiauthor},
  pdfcreator  = {XeLaTex},
  hidelinks,
}

\twocolumn[
  \vspace*{-5em}
  {Ausgabe Nr. \@multiausgabe} \hfill {\@multidate}\\
  {\includegraphics[width=\linewidth]{header/Ueberschrift}}%
  {\centering
  \hrule\vspace{\onelineskip}
  \@multilosung
  \vspace{\onelineskip}\hrule
  \vspace{\onelineskip}
  }\vspace*{2em}
]
}
%
%%%


% Termine und Impressum
%
% Use this command at the end of each file for nice looking dates. One item
% per date.
%
\newenvironment{termine}{%
  \begin{figure}[!b]
    \begin{framed}
      \textbf{Nächste Termine:} \par
      \begin{itemize}
}{%
      \end{itemize}
    \end{framed}
    \vspace*{-2mm}
    \begin{Spacing}{0.5}
    \tiny
    {\RaggedRight
    \begin{wrapfigure}{R}{.3cm}
    \vspace*{-.55cm}
    \hspace*{-.3cm}
    \begin{sideways}
    \doclicenseImage[imagewidth=1.5cm]
    \end{sideways}
    \end{wrapfigure}
    Disclaimer \& Impressum: Teile des Inhalts sind rein fiktional; Ähnlichkeiten mit realen Personen und Begebenheiten sind zufällig und nicht beabsichtigt.\\
    \smallskip
    V.i.S.d.P. Hanna Franzen, RPG Librarium Aachen e.V. (VR 5440) \\
    Kontakt: Postfach 101632, 52016 Aachen, \href{mailto:multiversum@rpg-librarium.de}{multiversum@rpg-librarium.de} \\
    \smallskip
    \doclicenseText
    \par}
    \end{Spacing}
  \end{figure}
}
%

% Author-Mark
\newcommand{\Verfasser}[2][]{%
\par{\raggedleft{}{\itshape \mbox{#2}%
\ifx\relax#1\relax%
\else%
~(\mbox{#1})%
\fi%
}\par}}

\newcommand{\verfasser}[2][]{%
\hfill{\itshape \mbox{#2}%
\ifx\relax#1\relax%
\else%
~(\mbox{#1})%
\fi%
}}
%

% Annonce-Mark
\newcommand{\zeichen}[1]{%
  \par{\raggedleft{}Zeichen: \MakeUppercase{#1}\par}%
}
%

\newcommand{\zeitung}[1]{%
  \par{\raggedleft{}\itshape{}#1\par}\noindent%
}%
%
%%%

\makeatother

\usepackage{enumitem}


%%%
% Set those variables

% Authors of the document.
% e.g. Max Mustermann, Erika Musterfrau
\multiauthor{Konstantin, Richard}

% Date of release.
% e.g. 31.12.2074
\multidate{Januar 2020}

% Number of release, no leading zeros.
% e.g. 15
\multiausgabe{18}

% Losung
% e.g. Die Kuh lief um den Teig.
\multilosung{Schwarze Magie brennt schlechter!}


% Logo
% Use a different logo. Defaults to Ueberschrift.svg
%\multilogo{Ueberschrift_xmas}

%
%%%

%%%%%%%%%%%%%%%%%%%%%%%%%%%%%%%%% DOCUMENT %%%%%%%%%%%%%%%%%%%%%%%%%%%%%%%%%
\begin{document}

\makemultititle
%

% PUT BODY HERE
\section{Was bisher geschah...}

\subsection{Freizeit}
Vor nich ganz einem Monde, so erzählt man sich, begab es sich zur Kirchen bei Sieg.
Die altehrwürde Freusburg, gestürmt wurde sie, von einer wilden Horde.
Magier, Trolle, Zwerde, blecherne Wesen aber auch leicht bekleidete Gestalten und sogar Tiere aus Textilfaser wurden gesichtet!

Über 5 Tage hinweg belagerten sie den Musiktum un plünderten das Buffet!
Erzählt wurde von fremden Welten.
Dabei geschmissen mit komischen Steinen.

Nach dem letzten Mahl sind sie sodann entschwunden, so schnell, wie sie gekommen warn.
Seither wurden sie nicht mehr erblickt, in jeden fernen Tal.
\Verfasser{Richard}

\subsection{Spaß mit Zahlen}
Von welchem System ist der Buchvorrat am größten?
Welches Buch das wertvollste?
Wie viele Werke sind neu hinzugekommen?
Was wollen wir noch haben?

Antworten auf diese Fragen gab es beim Treffen im Januar.
Neben einem Jahresrückblick und ansprechenden Visualisierungen vieler spannender Zahlen, wurde eine Wunschliste geschrieben fürs neue Jahr geschrieben.

Noch mehr Zahlen (in Form von Jahresabschluss und Finanzplan) erwarten Euch auch im Februar, denn dann ist es mal wieder Zeit für eine Mitgliederversammlung des Librariums.
Kommt vorbei, stürzt die Aristokratie, startet eine Revolution … oder trinkt einfach nur einen Kakao.\verfasser{Richard}


\section{An einem anderen Ort}


\subsection{Produktrückruf Besitzurkunde}
Die zur Wintersonnenwende von dem Schriftmeister Pairo (nom. script.) ausgestellte Urkunde wird zurückgerufen.
Die fehlerhafte Ausstellung ist dem vorzüglichen Wein der Katakombentaverne geschuldet.

Entgegen der Behauptung der Urkunde haben die Besitzenden der Urkunde nicht \textit{notwendigerweise} einen Besitzanspruch an allen Pflöcken der Welt.
Die Angabe, dass sie darüber hinaus Anrecht auf vier weitere Pflöcke hätten, ist ebenfalls fehlerhaft.
Der Autor maßt sich keine Aussage über die Rechtsgeschäfte der als \enquote{die Rauferei} bekannten Gruppe an.
Ihm ist insbesondere \textit{nicht} bekannt, dass diese Gruppe unehrenhaft handeln würde, Essen oder Kleidung gestohlen hätte oder einen illegalen Kampfring betreiben würde.
\verfasser[LARP]{Konstantin}


\subsection{Büromaterialbestellung -- Eilt!}
An: Zentrales Ressourcenmanagement \\
Von: Miesermeier, Zweigstelle Kirchen (Sieg) \\

\noindent
Liebe Kollegen der INSPECTRES Zentrale,

während der laufenden Einarbeitungszeit mussten wir leider feststellen, dass noch Ressourcen vor Ort benötigt werden.
Diese ergeben sich sowohl aus dem letzten Einsatz, als auch aus dem generellen Materialbedarf.
Daher beantrage ich hiermit die folgenden Ressourcen zu dienstlichen Zwecken:

\begin{enumerate}[leftmargin=*, itemsep=1ex]
  \item Ein Schreibtisch, bevorzugt Buche. \\
    Art der Beschaffung: Ersatz. (Für den defekten Tisch, der mir zur Einstellung zugeteilt wurde.) \\
    Begründung:
      Der vorhandene Schriebtisch kann nicht mehr als solcher bezeichnet werden.
      Er ist, vermutlich bedingt durch Materialfehler, schon am ersten Tag bei geringster Belastung zusammengebrochen.

  \item Sonderentsorgungscontainer. \\
    Art der Beschaffung: Temporär. (Logistik.) \\% zu Transport- bzw. Entsorgungszwecken. \\
    Begründung:
      Überreste einer Entität (vogelähnlich, Größenklasse 1-2m, mittels Axt zerstückelt) müssen zeitnah vom Burghof entfernt werden.
      Vermutlich übernatürlich, daher für die Standard-Abfallentsorgung ungeeignet.
      Kann Reste von Arzneiprodukten enthalten.
      (Siehe auch: Einsatzbericht vom 2020-01-02.)

  \item Personal. \\
    Art der Beschaffung: Ersatz. \\
    Begründung:
      Im Zuge des letzten Einsatzes ist Z4 gegangen und bisher nicht wieder zum Dienst erschienen.
      Es ist nicht mehr mit einer zeitnahmen Rückkehr zu rechnen.
      Wir benötigen personelle Verstärkung.
      (Diesmal bevorzugt nicht in kommende Fäll verwickelt.)

  \item Lärmschutzdämmaterial. \\
    Art der Beschaffung: Neubeschaffung, bauliche Maßnahme. \\
    Begründung:
      Unter der akuten Lärmbelastung durch die Musik des Kollegens Chopper lässt sich hier nicht mehr produktiv arbeiten.
      Eine Bauliche Maßnahme erscheint hier die sinnvollste Lösung zu sein.

  \item KFZ-Radio. \\
    Art der Beschaffung: Ersatz. \\
    Begründng:
      Das vorhandene Radio des Dinestpanzers kam in Kontakt mit einer Rohrzange und wurde dabei tragischerweise beschätdigt.
      Da es sich nun nicht mehr abschalten lässt und die Nachbar sich beschweren, ist schnellstmöglich Ersatz erforderlich.
      (Außerdem weigert sich der Kollege ohne Musik zu fahren, daher ist ein Ausbau ohne Ersatz keine Option.)
\end{enumerate}

\noindent
Mit besten Grüßen, \\
Miesermeier
\verfasser[Inspectres]{Richard}

\subsection{Titel 1}
Lorem Ipsum
\verfasser{Autor2}

\subsection{Titel 2}
Lorem Ipsum, text endet an Zeilenende.
\Verfasser[System]{Autor2}


\begin{termine}
% Put dates here:
  \item Mitgliederversammlung: 16.02.2020, 15~Uhr
  \item Monatliches Treffen: 16.02.2020, 19~Uhr
\end{termine}
\impressum

\end{document}
%%%%%%%%%%%%%%%%%%%%%%%%%%%%%%%%% END DOCUMENT %%%%%%%%%%%%%%%%%%%%%%%%%%%%%%%%%
