% !TeX TS-program = xelatex
% !TEX root = main.tex

% How to:
%


% Include preable with all definitions. Do not remove this line.
%%%
% Document Class: Memoir
% http://texdoc.net/texmf-dist/doc/latex/memoir/memman.pdf
\documentclass[a4paper,10pt,oneside,twocolumn,final,article]{memoir}
%
%%%

%%%
% variables
\makeatletter

\newcommand\multiauthor[1]{\renewcommand\@multiauthor{#1}}
\newcommand\@multiauthor{\@latex@error{No \noexpand\multiauthor given}\@ehc}

\newcommand\multidate[1]{\renewcommand\@multidate{#1}}
\newcommand\@multidate{\@latex@error{No \noexpand\multidate given}\@ehc}

\newcommand\multiausgabe[1]{\renewcommand\@multiausgabe{#1}}
\newcommand\@multiausgabe{\@latex@error{No \noexpand\multiausgabe given}\@ehc}

\newcommand\multilosung[1]{\renewcommand\@multilosung{\expandafter\MakeUppercase\expandafter{#1}}}
\newcommand\@multilosung{\@latex@error{No \noexpand\multilosung given}\@ehc}


%%%

%%%
% Fonts and Typesetting

% Mathspec: used to change math font.
\usepackage{mathspec}

%IPA
\usepackage{tipa}
\renewcommand\textipa[1]{{\fontfamily{cmr}\tipaencoding #1}}

%
% We use the open Adobe Source font family
% https://github.com/adobe-fonts
%
% Roman (Serif) Font: Source Serif Pro Light/Semibold
\setprimaryfont[
UprightFont={* Light},
ItalicFont={* Light Italic},
BoldFont={* Semibold},
BoldItalicFont={* Semibold Italic},
Scale=MatchLowercase]{Source Serif Pro}
%
% Sans Font: Source Sans Pro Light/Semibold
\setallsansfonts[
UprightFont={* Light},
ItalicFont={* Light Italic},
BoldFont={* Semibold},
BoldItalicFont={* Semibold Italic},
Scale=MatchLowercase]{Source Sans Pro}
%
% Mono Font: Source Code Pro Light/Semibold (No Italics available)
\setallmonofonts[
UprightFont={* Light},
BoldFont={* Semibold},
Scale=MatchLowercase]
{Source Code Pro}
%
% Replace math greek fonts by Source Serif Pro (No Italics available -> Use fake italics)
\setmathfont(Greek)[
UprightFont={* Light},
BoldFont={* Semibold},
AutoFakeSlant=0.15
]
{Source Serif Pro}
%
%%%

%%% Colors!
\usepackage[table]{xcolor}
%%%

%%%
% Memoir Class Page formatting
%
% Compute a suitable line width.
\setlxvchars \setxlvchars
% type block size:
\settypeblocksize{700pt}{1.6\lxvchars}{*}
% type block at the center of the page:
\setlrmargins{*}{*}{1}
\setulmargins{*}{*}{1}
% separator between two columns, no rule:
\setcolsepandrule{0.5cm}{0pt}
% header and footer height:
\setheadfoot{2\onelineskip}{2\onelineskip}
% note margin width and separator:
\setmarginnotes{17pt}{3.5cm}{\onelineskip}
% save changes and fix the layout:
\checkandfixthelayout
% empty page layout (no header, no footer)
\pagestyle{empty}
%
%%%


%%%
% Balance last two columns
% Set "balance" variable to "true" in pandoc to balance the columns on the last
% page.
%$if(balance)$\usepackage{flushend}$endif$
%%%


%%%
% Language
\usepackage[english,ngerman]{babel}
\usepackage[german=quotes]{csquotes}
%
%%%


%%%
% Figures
\usepackage{graphicx}
%
%%%

%%%
% Date Format:
\usepackage[ngerman]{datetime}
\newdateformat{monthyear}{\monthname[\THEMONTH] \THEYEAR}
%
%%%


%%%
% Licensing
\usepackage[
	type={CC},
	modifier={by-nc-sa},
	version={4.0},
]{doclicense}
\usepackage{rotating}
\usepackage{wrapfig,ragged2e}
%%%


%%%
% Meta data
\author{\@multiauthor}
\date{\@multidate}
\title{Multiversum Ausgabe Nr. \@multiausgabe}
%
%%%


%%%
% Cross-References, Links and PDF-Metadata
\usepackage{hyperref}
% Moved to makemultititle
% \hypersetup{
%	pdftitle    = {Multiversum Ausgabe Nr. \@multiausgabe},
%	pdfsubject  = {RPG Librarium Aachen e.V.},
%	pdfauthor   = {\@multiauthor},
%	pdfcreator  = {XeLaTex},
%	hidelinks,
%}
\urlstyle{sf}
%
%%%


%%%
% Divisions
% Section Style
\setbeforesecskip{-.5em}
\newcommand{\ruledsec}[1]{%
	\LARGE\scshape\centering #1 \rule{\linewidth}{0.4pt}\noindent}
\setsecheadstyle{\ruledsec}
\setaftersecskip{.5em}
% Subsection Style
\setsubsecheadstyle{\Large\normalfont\raggedright}
% Paragraph Style
\setparaheadstyle{\bfseries}
% Disable numbering
\setsecnumdepth{part}
% include subsections in PDF bookmarks
\setcounter{tocdepth}{2}
%
%%%


%%%
% Lists, Enumerations, Itemize
% tight spacing
\tightlists
%
%%%

%%%
% Cross out
\usepackage[normalem]{ulem}
%%%

%%%
% Macros
%

% Title
%
% Use this command to set a nice looking title.
% first argument: number, e.g. 15
% second argument: losung, e.g. Die Kuh lief um den Teich.
%
\newcommand{\makemultititle}{
\hypersetup{
  pdftitle    = {Multiversum Ausgabe Nr. \@multiausgabe},
  pdfsubject  = {RPG Librarium Aachen e.V.},
  pdfauthor   = {\@multiauthor},
  pdfcreator  = {XeLaTex},
  hidelinks,
}

\twocolumn[
  \vspace*{-5em}
  {Ausgabe Nr. \@multiausgabe} \hfill {\@multidate}\\
  {\includegraphics[width=\linewidth]{header/Ueberschrift}}%
  {\centering
  \hrule\vspace{\onelineskip}
  \@multilosung
  \vspace{\onelineskip}\hrule
  \vspace{\onelineskip}
  }\vspace*{2em}
]
}
%
%%%


% Termine und Impressum
%
% Use this command at the end of each file for nice looking dates. One item
% per date.
%
\newenvironment{termine}{%
  \begin{figure}[!b]
    \begin{framed}
      \textbf{Nächste Termine:} \par
      \begin{itemize}
}{%
      \end{itemize}
    \end{framed}
    \vspace*{-2mm}
    \begin{Spacing}{0.5}
    \tiny
    {\RaggedRight
    \begin{wrapfigure}{R}{.3cm}
    \vspace*{-.55cm}
    \hspace*{-.3cm}
    \begin{sideways}
    \doclicenseImage[imagewidth=1.5cm]
    \end{sideways}
    \end{wrapfigure}
    Disclaimer \& Impressum: Teile des Inhalts sind rein fiktional; Ähnlichkeiten mit realen Personen und Begebenheiten sind zufällig und nicht beabsichtigt.\\
    \smallskip
    V.i.S.d.P. Hanna Franzen, RPG Librarium Aachen e.V. (VR 5440) \\
    Kontakt: Postfach 101632, 52016 Aachen, \href{mailto:multiversum@rpg-librarium.de}{multiversum@rpg-librarium.de} \\
    \smallskip
    \doclicenseText
    \par}
    \end{Spacing}
  \end{figure}
}
%

% Author-Mark
\newcommand{\Verfasser}[2][]{%
\par{\raggedleft{}{\itshape \mbox{#2}%
\ifx\relax#1\relax%
\else%
~(\mbox{#1})%
\fi%
}\par}}

\newcommand{\verfasser}[2][]{%
\hfill{\itshape \mbox{#2}%
\ifx\relax#1\relax%
\else%
~(\mbox{#1})%
\fi%
}}
%

% Annonce-Mark
\newcommand{\zeichen}[1]{%
  \par{\raggedleft{}Zeichen: \MakeUppercase{#1}\par}%
}
%

\newcommand{\zeitung}[1]{%
  \par{\raggedleft{}\itshape{}#1\par}\noindent%
}%
%
%%%

\makeatother

%

%%%
% Set those variables

% Authors of the document.
% e.g. Max Mustermann, Erika Musterfrau
\multiauthor{Yoann, Richard, Hanna, Franca}

% Date of release.
% e.g. 31.12.2074
\multidate{Dezember 2019}

% Number of release, no leading zeros.
% e.g. 15
\multiausgabe{17}

% Losung
% e.g. Die Kuh lief um den Teig.
\multilosung{In diesem Falle, Riesenqualle}

%
%%%

%%%%%%%%%%%%%%%%%%%%%%%%%%%%%%%%% DOCUMENT %%%%%%%%%%%%%%%%%%%%%%%%%%%%%%%%%
\begin{document}

\makemultititle
%

% PUT BODY HERE
\section{Was bisher geschah...}

\subsection{Workshop: ABC}
Lorem Ipsum
\verfasser{Autor1}

\subsection{Spendenkampagne 2019}
Liebe Leserinnen und Leser, bitte verzeihen Sie die Störung. Es ist uns ein bisschen unangenehm, aber kommen wir gleich zur Sache. An diesem Tage sind Sie im Multiversum gefragt, um die Unabhängigkeit des RPG Librariums Aachen e.V. zu sichern:

Heute ist der 63. Tag unserer Bücherkaufkampagne. Der Librarium bekommt im Jahr durchschnittlich 3,5 Bücherwünsche, aber 99\% der Leserinnen und Leser schlagen nichts vor. Wenn alle Leser, die jetzt mitlesen, einen Bücherwunsch einsenden würden, wäre unsere Kampagne im Nu abgeschlossen.
Wir sind kein kommerzieller Verein, sondern ein Ort der Kunst, Kultur und natürlich der Ausgaben. Verschwenderische Ausgaben haben jedoch in unserem Verein nichts zu suchen. Im Gegensatz zu anderen großen Bibliotheken haben wir keine Personalkosten und auch unsere Hosting- und Verwaltungsgebühren belaufen sich nur auf einen Bruchteil unserer Einnahmen. Deshalb benötigen wir Ihre Hilfe. Um es dem Finanzamt recht zu machen, darf der Librarium auch in diesem Jahr keinen zu hohen Gewinn erwirtschaften.

Es ist leicht, diese Nachricht zu ignorieren und die meisten werden das wohl tun. Wenn Sie den Librarium nützlich finden, nehmen Sie sich an diesem Wochentag bitte eine Minute Zeit und geben dem Verein mit Ihrer Bücherwunschspende etwas zurück, damit der Librarium weiter bestehen kann. Vielen Dank!
\Verfasser{Redaktion}

\section{An einem anderen Ort}

\subsection{Strickschulden sind Ehrenschulden}
\enquote{So, \enquote{Bei Mittlerer Hitze 20 Minuten goldbraun backen.} Was soll das denn schon wieder heißen? Ach was soll's, 180°C werden schon passen.
Jetzt noch die Zeitschaltuhr\dots}

Erna Brügge wollte gerade nach der Eieruhr greifen, als plötzlich die Eingangstür zu ihrer kleinen Einzimmerwohnung mit einem lauten Knall explodierte. So schnell, wie es ihr gebrechlicher Körper erlaubte, wandte sie sich um. Durch die Trümmer der Haustür zwängte sich der mächtige Körper eines Trolls.

\enquote{Hallo Erna.}, begrüßte er sie mit rauchiger Stimme,
\enquote{Zweimal habe ich dir schon Aufschub gewährt. Jetzt ist es Zeitpunkt gekommen, an dem du lieferst.}

Mit einem panischen Blick in den Augen tat Erna einen Schritt zurück und stieß mit ihrem Rücken gegen die warm werdende Ofentür. \enquote{Ich hab sie noch nicht fertig!}, versuchte sie mit zittriger Stimme den Eindringling zu besänftigen, \enquote{Bitte gib mir noch was Zeit! Ich brauche nur eine weitere Woche\dots}

Jäh wurde sie von dem grollenden Troll unterbrochen.
\enquote{Nein! Ich will Sie jetzt. Ich habe dir Essen gegeben, als du am verhungern warst. Ich habe dir Unterschlupf gewährt, als es geregnet hat und ich habe dir Munition gegeben, als du deine letzte Patrone verschossen hattest. Warum?} Der Troll suchte nach den richtigen Worten. \enquote{Ich kann es dir nicht sagen. Wahrscheinlich aus Mitleid.}

Erna suchte verzweifelt einen Weg zum Fenster, durch das sie über die Feuerwehrleiter entkommen könnte, doch die breiten Schultern versperrten ihr jegliche Fluchtmöglichkeiten.
\enquote{Doch jetzt will ich endlich das, was du mir damals als Bezahlung versprochen hattest: Meine gestrickten Wollsocken.} Sein linkes Auge zuckte ein wenig. \enquote{Blaue kuschelige warme Wollsocken aus Merinowolle\dots}
Erna verlor die verlor die Kraft in ihren Beinen und rutschte an der der Ofentür zu Boden.

\enquote{Bitte, nur zwei Tage. Übermorgen kann ich liefern.}

\enquote{Deine Strickschulden sind dein Problem. Such dir jemand Anderes, dem du deine Lügen verkaufen kannst, aber jetzt will ich endlich was mir zusteht!}
\Verfasser[SR 5]{Yoann}

\subsection{Es begab sich aber zu der Zeit …}
... dass ein Auftrag von der Yakuza ausging, dass eine Lieferung Better-Than-Life-Chips nach Berlin zu schmuggeln sei.
Und diese Lieferung war nicht die erste und sie geschah im Jahre 2077, als Lofwyr CEO in Neu-Essen war.
Doch jeder Runner ging, da Heiligabend war, ein jeder in seine Stadt, und keiner wollte nach Berlin fahren.

Da machte sich ein Kurier auf aus dem Rhein-Ruhr-Plex, aus der Stadt Köln, durchs Rheinland zur Stadt der Yakuza, die da heißt Düsseldorf, weil er ein zuverlässiger Runner der Yakuza war und außerdem noch keine Pläne für Heiligabend hatte; damit er die BTLs abliefere mit seinem schnellen Auto, das da war ein Toyota.
Und als er Düsseldorf erreichte, kam die Zeit, dass er die BTLs verladen sollte.
Und er verlud die BTLs und wickelte sie in Geschenkpapier und legte sie in seinen Kofferraum, und oben drauf einen hässlichen Karl-Kombatmage-Strickpullover, denn man sollte die BTLs nicht finden.

Doch es waren Grenzschutzpolizisten auf der Grenze vom Rhein-Ruhr-Plex zum Kirchenstaat Westfalen, die hüteten des Nachts dort die Schlagbäume; und sie tranken heimlich Feuerzangenbowle, denn sie wären lieber zuhause bei ihren Familien gewesen.
Und der Kurier sah das und fuhr zu ihnen, und die Maria-Mercurial-Playlist aus seiner Musikanlage ertönte um sie, und sie wunderten sich sehr.
Und der Kurier sprach: \enquote{Wundert euch nicht! Seht, ich bringe euch Spekulatius und Zimtsterne als Weihnachtsgeschenk, und außerdem eine Summe von 500 Nuyen, damit ihr am heiligen Abend ein Auge zudrückt; denn ihr habt mich nicht gesehen!}
Und alsbald war bei dem Auto des Kuriers die Menge der beschwipsten Grenzschutzpolizisten, die lobten die Weihnachtsplätzchen und sprachen:
\enquote{Dank sei dem, der diese Plätzchen gebracht hat; und gesegnet sei das Weihnachtsfest dieses Halunken!}
Und als die Polizisten den Schlagbaum öffneten, sprach der Kurier zu sich:
\enquote{Wenn ich nun nach Berlin fahre, wird keiner die Geschichte glauben, die hier geschehen ist!}

Und er fuhr eilends und fand eine verborgene Route über dunkle Feldwege und erreichte Berlin.
Als er die Stadt aber erreichte hatte, lieferte er die BTL-Chips ab, und erhielt die vereinbarte Bezahlung und ein großzügiges Trinkgeld.
Den Karl-Kombatmage-Strickpullover aber behielt er, denn der hatte keine Blutflecken und war sehr kuschelig.
Und die, denen er von seiner Fahrt erzählte, wunderten sich sehr über das, was ihnen der Kurier berichtete.

\verfasser[SR 5]{Franca}


\subsection{Titel 2}
Lorem Ipsum, text endet an Zeilenende.
\Verfasser[System]{Autor2}


\begin{termine}
% Put dates here:
\item Termin: Termin: DD.MM.YY, hh Uhr
  \item Termin: DD.MM.YY - DD.MM.YY
\end{termine}

\end{document}
%%%%%%%%%%%%%%%%%%%%%%%%%%%%%%%%% END DOCUMENT %%%%%%%%%%%%%%%%%%%%%%%%%%%%%%%%%
