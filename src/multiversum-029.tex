% !TeX TS-program = xelatex
% !TEX root = main.tex

% How to:
%


% Load class with all definitions. Do not remove this line.
% Options will be passed to Memoir
\documentclass[final]{multiversum}
\usepackage[normalem]{ulem}
%

%%%
% Set those variables

% Authors of the document.
% e.g. Max Mustermann, Erika Musterfrau
\multiauthor{Faen, Franca, Jan}

% Date of release.
% e.g. 31.12.2074
\multidate{02.09.2024}

% Number of release, no leading zeros.
% e.g. 15
\multiausgabe{29}

% Losung
% e.g. Die Kuh lief um den Teig.
\multilosung{\emph{Mehr Tote für die Quote!} \textendash{} Leitspruch der Boron-Kirche}


% Logo
% Use a different logo. Defaults to Ueberschrift.svg
%\multilogo{Ueberschrift_xmas}

%
%%%

%%%%%%%%%%%%%%%%%%%%%%%%%%%%%%%%% DOCUMENT %%%%%%%%%%%%%%%%%%%%%%%%%%%%%%%%%
\begin{document}

\makemultititle
%

% PUT BODY HERE
%\section{Was bisher geschah...}

\subsection{Madalins Antwort auf Marcos Liebesbrief}
\sout{Wertester Marco}\\
\sout{Lieber Marco}\\
Hallo Marco,\\
deine Entschlossenheit, deine Gefühle für mich vor ganz Albenhus kundzutun, ist
beachtlich. Du veröffentlichtest deinen Liebesbrief, ja, gar einen Heiratsantrag
hier im Albenhuser Stadtanzeiger, wo jeder es lesen kann! Sei es der
Stadtgardist Seiler, der es in seiner verbalen Großzügigkeit liebt, jede
Neuigkeit schon am Stadttor zu teilen! Sei es der scharfzüngige Gildenmagier
Eichbald, der schon ein weises Urteil über deine Tat gefällt hat! Sei es gar
deine Mutter, die hier die intimsten Liebesbekenntnisse ihres Sohnes für die
Ewigkeit festgehalten liest! Jedermann in dieser Stadt vernahm deine Worte und
war -- so denke ich -- auf irgendeine Weise berührt.\\
Es beschämt mich, deiner Tollkühnheit nicht gleichzukommen, wenn ich dir per
Brief antworte. Mein Herz bricht, wenn ich dir nun mitteilen muss, dass meine
Verpflichtung gegenüber dem Geweihten Karlfried und seiner Schwester Arba (die
ebenfalls eine Geweihte ist, wenn du dich entsinnst!) mich in fremde Länder
treibt. Wer wäre ich, mich einer geweihten Mission zu verwehren?'\\
Trauere nicht: Unsere Sommerromanze wird zwar nicht zum Traviabund, doch die
Göttin Rahja lachte stets auf uns herab, und ich halte die Erinnerung in Ehre.

Phex sei mit dir und deiner Familie!\\
Madalin
(PS: Richte deiner sicher enttäuschten Mutter ein Garether Sprichwort
aus: Andere wohlhabende Familien haben auch schöne Töchter.)
\verfasser[DSA 4.1]{Franca}

\subsection{News von Donnerstag, 15. Mai 2081}
\textbf{SEA-News}

++Singapur: Seit mittlerweile zwei Wochen wird der bengalische Tiger Tapsi aus
dem Singapurer Zoo vermisst. Die 3 jährige Tigerdame wurde am Samstag morgen des
3. Mai nicht mehr in ihrem Gehege aufgefunden. Die Polizei vermutet, dass
Ökoterroristen sich in das Pflegepersonal des Zoos eingeschlichen und die
Tigerdame entführt haben, vermutlich mit dem Ziel sie wieder in ihre Heimat zu
bringen. Auf Hinweise hat die Polizei eine Belohnung ausgelobt. Sollten Sie
Tapsi sehen, nähern Sie sich nicht, sondern alarmieren sie sofort die Polizei!\\
\indent ++Sumatra: Shiawase gibt die Einstellung ihrer Abbautätigkeiten auf den Inseln
vor Sumatra bekannt. Die entstandenen Kosten ständen in keinem Verhältnis zum
Erzgehalt der Zinnsande, heißt es in einer Pressemitteilung. Die Maschinen
sollen nun in Minen in Ostafrika eingesetzt werden. Gerüchte, die
Abbauoperationen wurden von wilden Affen sabotiert, seien haltlos und erfunden.\\
\indent \textit{Design-A-Pet by Designer Genes: Erschaffen sie jetzt ihr eigenes
Haustier ganz nach ihren Wünschen. Ein Tigerbaby das nie wächst? Eine Katze mit
Flügeln? Ein Hamster der Tauchen kann? Alles ist möglich in unseren
Design-A-Pet-Kiosks! Lieferung innerhalb von 2 Wochen.}

\textbf{Pazifik-News:}

++ TVNZ Kidz: Die Fernsehsender von Aotearoa haben ein neues Kinderprogramm
angekündigt. Der seit vielen Jahren beliebte Goodnight Kiwi wird als
regelmäßiger Abendcartoon genauso auftauchen, wie Dokumentationen über Tiere und
eine Realityshow über Naga.\\
\indent ++ Melanesische Koalition: Der Flughafen Henderson Field ist erneut geschlossen
worden. Die dritte Schließung in den letzten zwei Jahren ist laut
Flughafenleitung erneut auf Geister zurückzuführen, die sich in der Nähe der
Landebahnen eingenistet haben. Wann der Flughafen wieder nutzbar sein wird,
ist noch nicht bekannt.\\
\indent \textit{Buzzy Bee AR: Jetzt neu, Buzzy Bee mit extra Features! Freunde für
Buzzy, ein Wald oder eine Blumenwiese. Ihr Kind wird niemals wieder eine anderes
Spielzeug wollen!}

\textbf{Internationales}

++Vereinigtes Königreich weist Botschafter der CAS aus: Die britische Regierung
hat bekannt gegeben, dass die Botschafter der konföderierten Staaten von Amerika
bis auf weiteres nicht mehr willkommen sind. Die Bekanntmachung folgt Gerüchten
über groß angelegte Razzias gegen Büros im zentralen London. Acht Personen
wurden wegen Verdachts auf Spionage festgenommen.\\
\indent ++UCAS weist Klage von Ares Macrotechnologies ab: Die Klage von Ares
Macrotechnologies gegen die Regierung der UCAS ist in der höchsten Instanz
gescheitert. Nach der Entziehung der Extraterritorialitsrechte im September
letzten Jahres hatte Ares eine Massenklage gegen die Regierung angestrebt. Diese
wurde nun vom Obersten Gerichtshof abgewiesen. Eine freie demokratische
Regierung müsste sich nicht den Wünschen von Firmen beugen, heißt es in der
Erklärung des Gerichtshofes.\\
\indent \textit{Besänftigung der Wilden Bestie: Lesen sie jetzt den neuen
Blogartikel des Drachen Rhonabwy über die bekannte Rockmusikerin Maria
Mercurial. Jeden Tag ein neuer Artikel oder ein neues Albumreview, nur mit dem
EVO-Musikabo. Nur 3,99¥ pro Woche!}
\verfasser[Shadowrun 5]{Jan}


\subsection{DSA: Ausfall eines Rollenspiel-System}
 Viele Rollenspieler*innen in meinem Umfeld haben irgendwann mal länger
\emph{Das Schwarze Auge DSA 4.1} gespielt. Oft waren es lange Kampanien und so
ganz lässt sie diese wundervoll ausgearbeitete Welt nicht mehr los. Doch wann
immer sie nochmal etwas anfangen  \textendash{} oder es versuchen, und kurz
danach wieder aufgeben  \textendash{} höre ich: \enquote{Ich spiele wieder
DSA\ldots trotz der Regeln.}

Für alle die DAS 4.1 nicht kennen hier ein kleines Beispiel aus dem Regelsystem,
um zu demonstrieren, wie toll *hust* dieses Regelsystem ist. Am besten lässt
sich das am \emph{Ausfall} zeigen, einem Kampfmanöver, bei dem der Gegner
zurückgedrängt werden soll. Klingt simpel, oder? Ich decke mein \enquote{Opfer}
so sehr mit Hieben und Stichen ein, dass es nur noch parieren und zurückweichen
kann\ldots{} wie viel Regeltext braucht man dafür? Schauen wir kurz ins \emph{Wege des
Schwerts} (kurz WdS), ein eigenes Buch, dass alle Kampfregeln abdeckt und finden
auf Seite~59-60\ldots *drum roll*\ldots 622~Worte (3593~Buchstaben). Das ist fast eine
Seite A4 in Calibri, Schriftgröße 11.

Was also ist laut DSA4.1 alles zu beachten bei \emph{Ich haue oft und schnell
zu}?  Zum einen müssen Paraden in Attacken umgewandelt werden für den Angriff
und Attacken in Paraden für die Verteidigung. Wie kompliziert ist das? WdS Seite
81 weiß die Antwort… in nur 454 Worten. Kurzantwort es erschwert die Kampfproben
und geht nicht mit alle Waffenarten gleich gut. Zum anderen müssen \emph{Freie
Aktionen} genutzt werden um Vorwärts (bzw Rückwärts bei der Verteidigung) zu
gehen. Was passiert, wenn die nicht habe oder nicht rückwärtsgehen kann? Das
wird direkt am Ort (also WdS Seite 60) erklärt. Was noch alles? Nun. Erst folgt
noch ein Abschnitt darüber mit welchen Waffen ich das kann und mit welchen
nicht. Wie viel \emph{Behinderung} (durch Rüstung und Gepäck) ich maximal haben
darf, um einen Ausfall machen zu dürfen. Es wird einzeln aufgeschlüsselt, wie
der Erste Angriff in der Angriffskette ist, was passiert, wenn nur wenig Platz
zur Verfügung steht, wenn ich noch besonders feste oder besonders geschickt
zuschlagen will, welche anderen Manöver ich wann noch machen darf, und wie es
mit der \emph{Distanzklasse} funktioniert, also wenn unterschiedlich lange
Waffen (zB Dolch gegen Speer) gekämpft wird oder was zu beachten ist, wenn der
Verteidiger ein Schild trägt. Zuletzt erfahre ich wie oft mein Charakter so
zuschlagen und wie so ein \emph{Ausfall} endet\ldots{} und ihr ahnt schon, liebe
Leser*innen, das da nicht steht \enquote{Wenn ihr nicht mehr draufhauen wollt,
endet der Ausfall}. Nein. Sieben Bullet-Points mit verschiedenen Optionen,
Proben, Erschwernissen und Folgen je nachdem ob beim Angriff verfehlt wird oder
beim Parieren versucht wird einfach stehen zu bleiben. Es gibt Verweise auf
\emph{Passierschläge} (WdS Seite 83, 372 Worte), Glückliche Paraden (WdS Seite
84, 104 Worte), das Manöver \emph{Gezieltes Ausweichen} (WdS Seite 66, 615
Worte)\ldots{} und nach gefühlt 2 Stunden Deep Dive fragen wir uns. Lohnt sich das
für meinen Charakter? Nach dem Prüfen von 4-6 Charakter und Waffenwerten kommen
wir zu dem Schluss: Manchmal.

Und dann, eines schönen Spielabends ist es so weit. Die perfekte Situation für
einen \emph{Ausfall} im Kampf stehen an. Euer Charakter hebt sein Schwert, ihr
hebt euer Wege des Schwerts - besser schnell nochmal nachlesen, ob irgendwas
übersehen wurde… aber alles passt. Euer Charakter fixiert das Ziel, ihr fixiert
euern Notizblock und rechnet die Erschwernisse und Erleichterung +4 hier, -2 da
und nochmal extra +4 weil ihr könnt und dadurch -4 für euren Gegner. Wenige
Minuten später. Der Würfel rollt! Das Schwert fährt nieder!\ldots{} und klatscht
nutzlos auf den Boden. War es ein schlechter Wurf? Hat der Gegner einfach eine
Mu-Probe geworfen und entschieden nicht zurückzuweichen? Habt ihr die Distanz-
oder Waffenklasse falsch eingeschätzt? Nein, Nein und nochmals Nein… Ihr habt
nur vorm Kampf vergessen euren Rucksack abzusetzen und damit ist die BE über dem
Grenzwert für dieses Manöver.  (Dieser Text hat 622 Worte und ist damit genau so
lang wie die Beschreibung des Manövers) 
\verfasser[DSA 4.1]{Faen}

\subsection{Noch ein Brief von Madalin}
Lieber Rodrunk,\\
wie habe ich mich über deine Briefe gefreut! Sagenhaft, dass du wahrlich mit dem
legendären Seefahrer Phileasson auf Reisen bist. Ich habe tatsächlich mal ein
Buch über diesen Phileasson gelesen, aber um ehrlich zu sein: Deine Briefe
übertreffen jeden Roman.  Offen gestanden: Ich gedenke demnächst, eine Reise
anzutreten, da es dir gelungen ist, meine Neugier zu wecken. Unter uns, die
Stadt Albenhus ist mir etwas zu eng und viel zu vertraut geworden. Ich sehne
mich nach der Freiheit einer neuen Reise und würde zu gern das Meer sehen. Wenn
ich richtig rechne, müsstest du bald wieder in Thorwal eintreffen, hoffentlich
als Sieger des Seefahrerwettstreitss? Rodrunk, König der Meere! Ich würde mich
äußerst freuen, dich in Thorwal wiederzutreffen. Vielleicht überzeuge ich noch
Karlfried und Arba, mich zu begleiten, doch mein baldiger Aufbruch ist
unvermeidlich.  Ich werde in den frühen Morgenstunden aufbrechen, um
tränenreichen Abschieden zu entgehen, und blicke mit Vorfreude auf neue
Abenteuer!

\noindent Hoffentlich bis bald in Thorwal,\\
Deine Madalin
\verfasser[DSA 4.1]{Franca}
\begin{termine}
% Put dates here:
  \item Monatstreffen: 16.09.24, 19 Uhr, Papillon
  \item SchwefeldrAachen Convention: \\21.09.24 \textendash{} 22.09.24, 
        Mefferdatisstraße 14-18
\end{termine}
\impressum

\end{document}
%%%%%%%%%%%%%%%%%%%%%%%%%%%%%%%%% END DOCUMENT %%%%%%%%%%%%%%%%%%%%%%%%%%%%%%%%%
