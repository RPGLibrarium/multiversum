% !TeX TS-program = xelatex
% !TEX root = main.tex

% How to:
%


% Load class with all definitions. Do not remove this line.
% Options will be passed to Memoir
\documentclass[final]{multiversum}
%

%%%
% Set those variables

% Authors of the document.
% e.g. Max Mustermann, Erika Musterfrau
\multiauthor{Hanna, Yoann}

% Date of release.
% e.g. 31.12.2074
\multidate{April 2020}

% Number of release, no leading zeros.
% e.g. 15
\multiausgabe{21}

% Losung
% e.g. Die Kuh lief um den Teig.
\multilosung{}


% Logo
% Use a different logo. Defaults to Ueberschrift.svg
%\multilogo{Ueberschrift_xmas}

%
%%%

%%%%%%%%%%%%%%%%%%%%%%%%%%%%%%%%% DOCUMENT %%%%%%%%%%%%%%%%%%%%%%%%%%%%%%%%%
\begin{document}

\makemultititle
%

% PUT BODY HERE
\section{Was bisher geschah...}

\subsection{Digitale, nicht beschlussfähige MV - wait, what?}
Inzwischen haben wir uns an unser neues, eingeigeltes Leben gewöhnt und die Fähigkeit verloren, anderen in die Augen zu schauen.
Das liegt an der Schwierigkeit, gleichzeitig in die Webcam und auf den Bildschirm zu schauen.
Entsprechend gerne würden wir eine echte Mitgliederversammlung besuchen und uns mal wieder ein Blickduell liefern. 
Leider ist das noch nicht möglich, daher war die MV am 27. Februar 2021 auch digital. 
Unpraktisch, da wir nur in einer persönlichen Zusammenkunft beschlussfähig sind.
Warum also überhaupt eine MV?
Die Antwort haben die Anwesenden recht schnell verstanden.
Wir konnten so schon einmal Einblick in alles wichtige bekommen:
\begin{itemize}
\item Trotz aller Widerigkeiten konnten 2020 11 (digitale) Monatstreffen mit zunehmender Professionalität stattfinden.
Was da passiert ist und ob ihr etwas verpasst habt erfahrt ihr in unserer \href{https://rpg-librarium.de/veranstaltungen/}{Veranstaltungsliste}.
\item Um den Mitgliedern weiter den Zugang zu ermöglichen, steht der Bücherschrank bis auf weiteres bei Franca. 
Angeblich wird er von einem Drachen bewacht, aber auf Anfrage erkämpft euch Franca ein Buch. 
Die Adresse der Drachenhöhle gibt es über eine Mail an \href{mailto:vorstand@rpg-librarium.de}{vorstand@rpg-librarium.de}.
\item Auch wenn der Jahresabschluss nicht beschlossen wurde, konnte der Vorstand Einblicke geben. 
Bei Interesse könnt ihr diese Infos auch vom \href{mailto:vorstand@rpg-librarium.de}{Vorstand} bekommen.
\item Da die Vorstandswahl verschoben werden musste, gab es einige Meinungsbilder - was wollen wir dieses Jahr tun? 
Sollen wir eine Freizeit 2022 in Erwägung ziehen? 
Unser bisheriger Vorstand gibt sich alle Mühe, bei nächster Gelegenheit nicht abgewählt zu werden. 
Wenn ihr nicht dabei wart und etwas ergänzen wollt, meldet euch beim \href{mailto:vorstand@rpg-librarium.de}{Vorstand}!
\item Natürlich haben wir auch Bücher- und diesmal Landkartenwünsche gesammelt.
Die könnt ihr natürlich jederzeit äußern. Schreib auch hierfür an den \href{mailto:vorstand@rpg-librarium.de}{Vorstand}.
\end{itemize}
Irgendwann, wenn wir uns wieder aus unseren Höhlen wagen können, bekommen wir wieder eine richtige MV mit Blackj\dots äh, nein, beschlussfähig natürlich. 
Bis dahin findet ihr alle wichtigen Updates auf der \href{https://rpg-librarium.de/}{Webseite} und im \href{https://lists.rpg-librarium.de/postorius/lists/newsletter.lists.rpg-librarium.de/}{Newsletter}.
\verfasser{Hanna}

\subsection{In der Redaktion}
Lorem Ipsum
\verfasser{Redaktion}

\section{An einem anderen Ort}

\subsection{Titel 1}
Lorem Ipsum
\verfasser{Autor2}

\subsection{Titel 2}
Lorem Ipsum, Text endet an Zeilenende.
\Verfasser[System]{Autor2}


\begin{termine}
% Put dates here:
\item Termin: Termin: DD.MM.YY, hh Uhr
  \item Termin: DD.MM.YY - DD.MM.YY
\end{termine}
\impressum

\end{document}
%%%%%%%%%%%%%%%%%%%%%%%%%%%%%%%%% END DOCUMENT %%%%%%%%%%%%%%%%%%%%%%%%%%%%%%%%%
