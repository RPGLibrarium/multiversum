% !TeX TS-program = xelatex
% !TEX root = main.tex

% How to:
%


% Load class with all definitions. Do not remove this line.
% Options will be passed to Memoir
\documentclass[final]{multiversum}
%

%%%
% Set those variables

% Authors of the document.
% e.g. Max Mustermann, Erika Musterfrau
\multiauthor{Hanna, Thibaud}

% Date of release.
% e.g. 31.12.2074
\multidate{XX.01.2025}

% Number of release, no leading zeros.
% e.g. 15
\multiausgabe{31}

% Losung
% e.g. Die Kuh lief um den Teig.
\multilosung{Gelb kam zu kurz}


% Logo
% Use a different logo. Defaults to Ueberschrift.svg
%\multilogo{Ueberschrift_xmas}

%
%%%

%%%%%%%%%%%%%%%%%%%%%%%%%%%%%%%%% DOCUMENT %%%%%%%%%%%%%%%%%%%%%%%%%%%%%%%%%
\begin{document}

\makemultititle
%

% PUT BODY HERE
\section{Was bisher geschah...}

\subsection{Weltenbrücken}
Auf dem ein oder anderen Treffen ist bereits das Wort \enquote{Weltenbrücken} gefallen.
Die Welten werden einmal im Jahr zusammen überbrückt, üblicherweise in der ersten Januarwoche -- genau, auf der Freizeit des Librariums auf der Freusburg im Siegerland.
Ursprünglich mit wenigen Teilnehmern ausprobiert, hat sich \enquote{Weltenbrücken} zur Tradition entwickelt.
Dabei spielen alle Teilnehmer parallel in unterschiedlichen Systemen.

Zunächst lernen die Spieler ihr System kennen und spielen eine Einleitung.
In diesem Jahr sind dabei die Apokalyptischen Reiter freigesetzt worden.
Die sehen in Mausritter natürlich anders aus als in Vampire, DnD oder Shadowrun.
In jeder Welt gibt es nun eine große Quest, um die drohende Apokalypse abzuwenden.
Ob Formular A38 uns dabei helfen kann? Wenn man es doch nur genehmigen lassen könnte!
Doch mit der drohenden Apokalypse werden auch die Barrieren zwischen den Welten instabiler.
Die Charaktere werden aus ihrer Welt gesogen und landen in einer ganz Anderen!
Welche Rolle spielt der Shadowrun-Charakter in einem Märchen?
Wie kommt ein Cthulhu-Charakter im Nachtclub der 2080er klar?
Was macht ein Vampir, wenn er plötzlich vor einer menschengroßen Maus steht?
Und der Druide hat mein Formular gefressen!

In einem heillosen, aber geordneten Chaos verfolgen die Charaktere ein gemeinsames Ziel: den Erhalt der Welten!
Dabei reisen sie durch dieselbigen und finden doch immer die gleiche Aufgabe vor, treffen bekannte Gesichter wieder und lernen neue Mitstreiter kennen.
Mit der letztendlichen Rettung der Welten -- mal mehr, mal weniger erfolgreich -- können die Charaktere schließlich in ihre eigenen Welten zurückkehren.
Der ein oder andere Charakter nimmt neue Eindrücke, exotische Gegenstände, neue Freunde oder tiefe Traumata mit.
Doch nicht jede Welt ist so, wie sie war.
\verfasser{Hanna}

\subsection{Hurra, hurra, die Pest ist da!}
Wir zählen das Jahr 1347. 
Der schwarze Tod wütet im Land und auch ihr seid befallen.
Doch entgegen aller Erwartungen scheint Gott noch Gnade walten zu lassen.
Zwar siecht ihr schon lange, aber niemand aus eurer Gruppe ist bisher verstorben.
So zieht ihr eures Weges, eine abgerissene Gruppe, die von Siegen zur großen Stadt Colonia zieht.
Euer Handkarren ist schwer, es regnet seit Tagen und der Winter naht.
Ein einsamer Wanderer warnt euch, dass etwas Dunkles über dem Landstrich liegt.
Die Not, sowie ein merkwürdiger Traum, treiben euch dennoch in das abweisende Herkersdorf.

Das System Memento Mori spielt mit dem Tod und der Verzweiflung des Spätmittelalters.
Gespielt werden Gezeichnete, die die Aufmerksamkeit von dunklen Mächten auf sich ziehen.
Zunächst noch menschlich und von der Pest getrieben entwickeln die Charaktere übernatürliche Kräfte, doch die Kosten sind hoch.
Während der Schleier sich lüftet, werden die alten Sagen lebendig und haben ihr morbides Interesse an den Gezeichneten.
Der Teufel am Wegesrand ist genauso echt wie die Kälte und die Pestbeulen.
Ein System, das langsam in einer harten Lebensrealität beginnt und in die dunklen Bereiche des Übernatürlichen eskaliert.
Memento Mori ist ein ernstes, stimmungsvolles Spiel, das in einer wunderschön gestalteten Box kommt.
\verfasser{Hanna}

\section{An einem anderen Ort}

\subsection{Titel 1}
Lorem Ipsum
\verfasser{Autor2}

\subsection{Titel 2}
Lorem Ipsum, Text endet an Zeilenende.
\Verfasser[System]{Autor2}

\section{Werbung}
\subsection{Freddis Fröhlichtrunk}
Gnädiges Fräulein! Gnädiger Herr!  

Leiden Sie an Schnupfen oder Melancholie? Fehlt ihnen die Manneskraft, oder schmerzt ihr Kreuz in der Nacht? Erwerben Sie hier und heute bei ihrem Meister-Apothekarius Frederikus

\begin{center}\textsc{Freddis Fröhlichtrunk}\end{center}

Es gibt nichts, was dieses Elexir nicht heilen kann! Erwerben Sie nur heute für zwei Thaler drei Flaschen des besonderen Mittels aus über hundert Heilkräutern. 
\Verfasser[Memento Mori]{Thibaud}


\begin{termine}
% Put dates here:
\item Termin: Termin: DD.MM.YY, hh Uhr
  \item Termin: DD.MM.YY - DD.MM.YY
\end{termine}
\impressum

\end{document}
%%%%%%%%%%%%%%%%%%%%%%%%%%%%%%%%% END DOCUMENT %%%%%%%%%%%%%%%%%%%%%%%%%%%%%%%%%
