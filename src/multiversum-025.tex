% !TeX TS-program = xelatex
% !TEX root = main.tex

% How to:
%


% Load class with all definitions. Do not remove this line.
% Options will be passed to Memoir
\documentclass[final]{multiversum}
%

%%%
% Set those variables

% Authors of the document.
% e.g. Max Mustermann, Erika Musterfrau
\multiauthor{}

% Date of release.
% e.g. 31.12.2074
\multidate{}

% Number of release, no leading zeros.
% e.g. 15
\multiausgabe{25}

% Losung
% e.g. Die Kuh lief um den Teig.
\multilosung{Nachtsicht ist besser als Vorsicht}


% Logo
% Use a different logo. Defaults to Ueberschrift.svg
%\multilogo{Ueberschrift_xmas}

%
%%%

%%%%%%%%%%%%%%%%%%%%%%%%%%%%%%%%% DOCUMENT %%%%%%%%%%%%%%%%%%%%%%%%%%%%%%%%%
\begin{document}

\makemultititle
%

% PUT BODY HERE

\section{Was noch geschieht...}

\subsection{Multiversum Plus Maximus Premium}
Auch wir müssen unsere Kosten decken und wir sprechen uns deutlich gegen die Gratiskultur aus, die sich im Multiversum etabliert hat. Uploadfilter, Redakteure und Server müssen bezahlt werden, ganz zu schweigen von der hohen Kapitalertragssteuer, die uns jährlich zu schaffen macht.

Ausgewählte Ausgaben sind in Zukunft ausschließlich zahlenden Mitgliedern vorbehalten.
\begin{center}Introducing \textsc{Multiversum Plus Maximus Premium!}\end{center}

Selbstverständlich legen wir weiterhin die gewohnt hohen Qualitätsmaßstäbe an unsere kostenlosen Ausgaben an.
Es wird weiterhin regelmäßig zu jedem 1. eines jeden Monats eine gebührenfreie Ausgabe erscheinen.
Desweiteren wird weiter wie gewohnt Qualtitätswerbung...

Blurring out...
\verfasser{Redaktion}

\section{Was bisher geschah...}

\subsection{Enttäuschung im Literatursammlungspreis}
Wie ihr sicher alle wisst, ist der Librarium dieses Jahr für den Deutschen Umverteilungs-, Mitsprache-, und Bibliothekspreis (DUMB) nominiert.
Darin werden Vereinigungen gekührt, die Literatur sammeln.
Mit unseren 246 Titeln des Rollenspiels ist uns eine Nominierung in der Rubrik \enquote{Verteilte Klein-Bibliotheken eingetragener gemeinnütziger Vereine} sicher.
Leider müssen wir euch mitteilen, dass wir den Preis dieses Jahr nicht bekommen haben.
Der Preisträger ist stattdessen der Kleingartenverein "Karottensammler e.V.", die an jeden Mieter ihrer 250 Parzellen die gebundene Informationsbroschüre "Karotten pflanzen leicht gemacht" verteilen.
Unsere Vorsitzende Franca sagt dazu: \enquote{}.
Wir können ihr nur zustimmen.
\verfasser{Hanna}

\subsection{In der Redaktion}
Lorem Ipsum
\verfasser{Redaktion}

\section{An einem anderen Ort}

\subsection{Falscher Praiot enttarnt}
\zeitung{Der Marktschreier}
\textit{Albenhus.} 
Skandal, Skandal!
Viele Reisende nutzen auf dem Weg von und nach Albenhus den kleinen, schäbigen Praiosschrein.
Der zuständige Praiot, Praioslieb, ist falsch!
Im Auftrag der Gräfin wurde er enttarnt und der Stadtwache ausgeliefert!
Skandal, Skandal!
Der anmaßende falsche Priester wird des Nachmittags verbrannt so Praios will, jawohl!
Das heilige Auge des Praios wird zur Praiosstunde auf ihn niederschauen!
Ein echter Praiot - ja, wir sind uns diesmal wirklich sicher, extra von den anderen Priestern der Stadt bestätigt - ein echter Praiot wird kommen, um den falschen Praiot zu richten!
Praios Gerechtigkeit wird ihn treffen!
Wir erwarten eine schöne Verbrennung auf dem Marktplatz.
Schiebt eure Karren vor der Praiosstunde zur Seite, um die Errichtung der Richtplattform zu ermöglichen!
Macht Plaaatz!
Baut die Fressstände da hinten auf und du da mit dem Bier, du weiter nach rechts!
Skandal, Skandal!
Das wird eine schöne Verbrennung, bringt eure Kinder!
Skandal, Skandal!
\verfasser[DSA 4.1]{Hanna}

\subsection{Das Wunder von Albenhus}
\zeitung{Der Kurier}
\textit{Zollfeste nahe Albenhus.} 
\verfasser[DSA 4.1]{Hanna}

\subsection{Titel 2}
Lorem Ipsum, Text endet an Zeilenende.
\Verfasser[System]{Autor2}


\begin{termine}
% Put dates here:
\item Termin: Termin: DD.MM.YY, hh Uhr
  \item Termin: DD.MM.YY - DD.MM.YY
\end{termine}
\impressum

\end{document}
%%%%%%%%%%%%%%%%%%%%%%%%%%%%%%%%% END DOCUMENT %%%%%%%%%%%%%%%%%%%%%%%%%%%%%%%%%
