% !TeX TS-program = xelatex
% !TEX root = main.tex

% How to:
%


% Load class with all definitions. Do not remove this line.
% Options will be passed to Memoir
\documentclass[final]{multiversum}
%

%%%
% Set those variables

% Authors of the document.
% e.g. Max Mustermann, Erika Musterfrau
\multiauthor{Yoann, Hanna, Franca, Konstantin}

% Date of release.
% e.g. 31.12.2074
\multidate{April 2022}

% Number of release, no leading zeros.
% e.g. 15
\multiausgabe{25}

% Losung
% e.g. Die Kuh lief um den Teig.
\multilosung{Nachtsicht ist besser als Vorsicht}


% Logo
% Use a different logo. Defaults to Ueberschrift.svg
%\multilogo{Ueberschrift_xmas}

%
%%%

%%%%%%%%%%%%%%%%%%%%%%%%%%%%%%%%% DOCUMENT %%%%%%%%%%%%%%%%%%%%%%%%%%%%%%%%%
\begin{document}

\makemultititle
%

% PUT BODY HERE

\section{Was noch geschieht...}

\subsection{Multiversum Plus Maximus Premium}
Auch wir müssen unsere Kosten decken und wir sprechen uns deutlich gegen die Gratiskultur aus, die sich im Multiversum etabliert hat. 
Uploadfilter, Redakteure und Server müssen bezahlt werden, ganz zu schweigen von den hohen Gebühren des Transparenzregisters, die uns jährlich zu schaffen machen.

Ausgewählte Ausgaben sind in Zukunft ausschließlich zahlenden Mitgliedern vorbehalten.
\begin{center}Introducing \textsc{Multiversum Plus Maximus Premium!}\end{center}

Selbstverständlich legen wir weiterhin die gewohnt hohen Qualitätsmaßstäbe an unsere kostenlosen Ausgaben an.
Es wird weiterhin regelmäßig zu jedem 1. eines jeden Monats eine gebührenfreie Ausgabe erscheinen.
Desweiteren wird weiter wie gewohnt Qualitätswerbung...

Blurring out...
\verfasser{Redaktion}

\section{Was bisher geschah...}

\subsection{Enttäuschung im Literatursammlungspreis}
Wie ihr sicher alle wisst, ist der Librarium dieses Jahr für den Deutschen Umverteilungs-, Mitsprache-, und Bibliothekspreis (DUMB) nominiert.
Darin werden Vereinigungen gekürt, die Literatur sammeln.
Ebenfalls nominiert ist der Librarium für den etwas kleineren Preis der Kleinbibliotheken eingetragener Reorganisationen (KleBeR).

Mit unseren 246 Titeln des Rollenspiels ist uns eine Nominierung in der DUMB Rubrik \enquote{Verteilte Klein-Bibliotheken eingetragener gemeinnütziger Vereine} sicher.
Leider müssen wir euch mitteilen, dass wir den Preis dieses Jahr nicht bekommen haben.
Der Preisträger ist stattdessen der Kleingartenverein "Karottensammler e.V.", die an jeden Mieter ihrer 250 Parzellen die gebundene Informationsbroschüre \enquote{Karotten pflanzen leicht gemacht} verteilen.
Unsere Vorsitzende Franca sagt dazu: \enquote{Ich gratuliere dem Karottensammler e.V. zu ihrem wohlverdienten Sieg. 
Wie sagt man so schön? Die Karotte ist mächtiger als das Schwert. Und auch besser für die Augen}.
Wir können ihr nur zustimmen.

\hyphenation{Nie-der-un-ter-teu-peln}
Unklar ist noch die Preisverleihung des KleBeRs, da der eigentlich geplante Ort der Preisverleihung, die Sporthalle der Gemeinschaftsgrundschule Niederunterteupeln, zur Zeit wegen eines Coronafalls geschlossen bleiben muss.
Wir hoffen auf das Beste und freuen uns auf die Ersatzveranstaltung, die vorraussichtlich auf April 2024 verschoben wird.
\verfasser{Hanna}

\section{In der Redaktion}
Uff. Wieder ein Morgen in der Redaktion.
Die Kaffeemaschine läuft schon und die Redakteure sitzen mit Tassen in den Händen vor den Rechnern.
Ich habe bereits den Eingang sortiert: eine Email.
Die wurde in der Nacht schon automatisch ausgedruckt und in den Eingangskorb geleitet.
Darin wurde gefragt, ob ich Viagra von einem reichen afrikanischen Prinzen kaufen will.
Das war eher suggestiv, daher ist dieser Redaktionsfetischist im Papierkorb gelandet.
Auf der Arbeit ist das einfach nicht angebracht.

Der Chef kommt mit seiner Tasse in der Hand herangeschlurft, er hat sich eben erst den ersten Kaffee geholt.
\enquote{Sag mal, du hast bestimmt ein bisschen Zeit?}
Oh, warum ist er denn heute so nett?
In den letzten Wochen habe ich für ihn einige Recherchen gemacht, die bereits vor zwei Stunden fertig sein sollten, als er sie mir übertragen hat.
Das liegt bestimmt am mangelnden Kaffee.

Als ich nicke, fährt er fort:
\enquote{Du hast vielleicht bemerkt, dass der Besprechungsraum immer noch etwas unaufgeräumt ist.}

Meint er den Besprechungsraum mit dem Dreifuß und der Linse?
Den, den die Putzkräfte meiden, weil der Staub zu hoch liegt und die Glaswände immer noch mit Tinte beschmiert sind?

 \enquote{Ich dachte, da könntest du mal etwas Ordnung reinbringen.}

 Ich schaue ihn unverhohlen entsetzt an.
 \enquote{Gibt es mehr als einen Besprechungsraum?}, frage ich vorsichtig.

 \enquote{Nein, warum? Du kennst den Raum doch? Am Ende des Gangs?}
 Ich seufze stark.
 Das nächste Praktikum lasse ich mir bezahlen, nehme ich mir vor.
\verfasser{Die Praktikantin}

\pagebreak
\section{An einem anderen Ort}

\subsection{Falscher Praiot enttarnt}
\zeitung{Der Marktschreier}
\textit{Albenhus.} 
Skandal, Skandal!
Viele Reisende nutzen auf dem Weg von und nach Albenhus den kleinen, schäbigen Praiosschrein.
Der zuständige Praiot, Praioslieb, ist falsch!
Im Auftrag der Gräfin wurde er enttarnt und der Stadtwache ausgeliefert!
Skandal, Skandal!
Der anmaßende falsche Priester wird des Nachmittags verbrannt so Praios will, jawohl!
Das heilige Auge des Praios wird zur Praiosstunde auf ihn niederschauen!
Ein echter Praiot - ja, wir sind uns diesmal wirklich sicher, extra von den anderen Priestern der Stadt bestätigt - ein echter Praiot wird kommen, um den falschen Praiot zu richten!
Praios Gerechtigkeit wird ihn treffen!
Wir erwarten eine schöne Verbrennung auf dem Marktplatz.
Schiebt eure Karren vor der Praiosstunde zur Seite, um die Errichtung der Richtplattform zu ermöglichen!
Macht Plaaatz!
Baut die Fressstände da hinten auf und du da mit dem Bier, du weiter nach rechts!
Skandal, Skandal!
Das wird eine schöne Verbrennung, bringt eure Kinder!
Skandal, Skandal!
\vspace{1em}
\zeitung{Der Kurier}
\textit{Zollfeste nahe Albenhus. 2 Stunden später.}\\
\enquote{Die Stadt brennt! Albenhus brennt!}
\Verfasser[DSA 4.1]{Hanna}

\subsection{Staubsaugerdroide verschlingt Coruscant und erklärt sich zum neuen Imperator}

\textit{Coruscant-System.} 
In der Nacht von Donnerstag auf Freitag hat ein Staubsaugerdroide den Planeten Coruscant verschlungen. 
Der nicht-lizenzierte Droide \enquote{Maw 2.0} näherte sich dem galaktischen Regierungshauptsitz über die hydianische Hyperraumroute und saugte innerhalb kürzester Zeit Gebäude, Bewohner und Landmasse des Planeten auf. 
Dabei ging \enquote{Maw 2.0} derart subtil vor, dass das Fehlen von Coruscant erst Stunden später von einem einzelnen imperialen TIE-Fighter-Piloten bemerkt wurde, der vergeblich seinen Hangar suchte. 

Kurze Zeit später sendete \enquote{Maw 2.0} eine galaxisweite Holo-Nachricht aus dem Coruscant-System. 
Die Nachricht bestand aus Piepsen, Brummen und schlürfenden Sauggeräuschen, welche von Zeugen durchweg als „bedrohlich“ bezeichnet wurden. 
Dabei trug der Staubsaugerdroide eine schwarze Kapuze und einen langen dunklen Umhang. 
Experten übersetzen die Botschaft mit \enquote{Ich bin jetzt der Boss. 
Benehmt euch, sonst sauge ich euch weg! Lang lebe Imperator Maw!}

Die Rebellion leugnet ihre Beteiligung: \enquote{Einen ganzen Planeten einfach verschwinden lassen? 
Wer käme denn auf die bescheuerte Idee? 
Ja, ja, wir sind immer noch grantig wegen der Alderaan-Nummer. 
Aber dieser planetare Frühjahrsputz geht nicht auf unser Konto! 
Schließlich sind die Hälfte unserer X-Wing-Piloten Deserteure von den Coruscanter Akademien. 
Dieser Fachkräftemangel wird uns bald hart treffen.}

Auch im Outer Rim zeigt man sich fassungslos. 
\enquote{Der Untergrund von Coruscant war unser bester Kunde! 
Der ganze Abschaum der Gesellschaft, einfach weg! 
Wer kauft denn jetzt unser Spice?}, klagt das Pyke-Syndikat und stellt zugleich eine Vermutung auf: 
\enquote{Wir glauben, das Schwarze-Sonne-Kartell steht hinter Imperator Maw und sie sabotieren gezielt unsere Geschäfte im galaktischen Kern. 
Die Schwarze Sonne hat bekanntlich sogar den ehrbaren Hutten Kalto abgemurkst. 
Die Widerlinge schrecken vor nichts zurück.}

Die Willkürherrschaft eines Staubsaugerdroiden stellt für manche Planeten eine existenzbedrohende Gefahr da. 
Tatooine und Ryloth riefen zum solidarischen Zusammenhalt verödeter Sandplaneten auf: 
\enquote{Coruscant ist weit entfernt. 
Viele Parsec voller Asteroidenstaub trennen uns von Imperator Maw. 
Aber wenn uns irgendwer unseren geliebten Schmutz streitig machen will, dann stehen wir zusammen und lassen die Sarlaccs los! 
Wollen mal sehen, wer wen zuerst vertilgt!}

Auf Nal Hutta dagegen ist die Verunsicherung groß. 
\enquote{Bei uns ist es ziemlich schlammig. 
Momentan bemühen wir uns um eine Einschätzung, ob Maw auch feucht wischen kann. 
Wenn ja, dann haben wir ein ernsthaftes Problem.}

Die nächsten Schritte von Imperator Maw sind noch nicht abzusehen. 
Gerüchte besagen aber, dass Imperator Maw momentan sein Regierungskabinett aufstellt und Ministerposten an seine engsten Vertrauten vergibt. 
Die Rede ist von mehreren neugeschaffenen Ministerien; unter anderem einem Ministerium für die grundlegende Reinigung der Dunklen Macht, einem Hyperreinraum-Ministerium für saubere Weltraumrouten und einem Ministerium für streifenfreies Fensterputzen unter Vakuumbedingungen.
\verfasser[Star Wars]{Franca}

\section{Werbung}

Haben Sie es satt, Ihre Kameraden nicht jederzeit an Ihre Göttin der Dunkelheit opfern zu können? 
Damit ist jetzt Schluss! Der Paktinator 3000 erlaubt Ihnen, zum niedrigen Preis von einer halben Seele\footnote{Zerreißen der Seele fällt in die Verantwortung des Kunden; zusätzliche AGB finden Anwendung.} einen persönlichen Begleitdämon zu erwerben, der Ihre Kameraden einfach regelmäßig verschlingt! Sie können ruhig schlafen, während Ihre treuen Begleiter geopfert werden - ohne dass Sie einen Finger krümmen müssen! 

Weitere Informationen erhalten Sie in Ihrem nächsten Shar-Tempel oder in dem nächsten Alptraum. 
\Verfasser[DnD]{Konstantin}

\begin{termine}
% Put dates here:
\item Monatstreffen: 16.04.22, 19 Uhr
\end{termine}
\impressum

\end{document}
%%%%%%%%%%%%%%%%%%%%%%%%%%%%%%%%% END DOCUMENT %%%%%%%%%%%%%%%%%%%%%%%%%%%%%%%%%
